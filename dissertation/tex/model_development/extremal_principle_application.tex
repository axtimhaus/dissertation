\section{Application of the Thermodynamic Extremal Principle (TEP)}\label{sec:extremal-principle-application}

In the following the generalized extremal principle is applied on the current model of particles and nodes, see \autoref{subsubsec:extremal-priciple-generalized} for details on the approach.
As before, an index of the currently regarded particle or node is neglected where appropriate for brevity.
The indizes $\Upper\square$ and $\Lower\square$ are used to denote the upper resp.~lower neighbor of a node.

\subsection{Time Scale Definition}\label{subsec:time-scale-definition}

The application of the TEP results in ordinary differential equations in time which can be solved analytically for simple problems.
Here, however, the problem is to complex to be solved analytically, so numerical time integration is required.
Usually, the transition from infinitesimal description to finite time steps would be performed as a last step, but doing this step at the beginning opens possibilities of simplifying the equations and clarifies the following elaborations.
At this point, it does not matter if a simple Eulerian integration is used or more complex methods like Runge-Kutta.
The key feature is, that we use a finite time step width $\TimeStep$, which is fixed in each time step calculation, but may change from step to step.
So, the internal state velocities $\InternalStateVelocity$ are directly linked linearly to finite changes using the time step as follows:
\begin{equation}
    \InternalStateStep = \InternalStateVelocity \cdot \TimeStep
\end{equation}

\subsection{Choice of Variables}\label{subsec:choice-of-variables}

First, the internal state variables $\vect\InternalStateVariable$, the internal state velocities $\vect\InternalStateVelocity$ and the fluxes $\vect\Flux$ must be chosen.
The feasibility of the approach heavily depends on this choice.

The choice of the fluxes is straightforward with the diffusional fluxes.
One has two diffusional fluxes on each node $\Upper\Flux$ and $\Lower\Flux$, however the flux of the upper node to the lower and the flux from the lower to the upper are always equal due to constance of mass.
Additional fluxes are to be found in the fluxes from the particle to the matrix $\Matrix\Flux$, if a matrix is present in the inter-particle spaces.

The internal state variables are the coordinates of the particles $\left[ \Particle\Radius, \Particle\Angle, \Particle\RotationAngle \right]$ and their nodes $\left[ \Radius, \Angle \right]$ in space, since they determine the thermodynamic forces and the main aim of the simulation is to follow the time evolution of the particles and nodes.

The straightforward choice of the internal state velocities are the finite changes of internal state variables (e.g. $\Delta\Radius$ and $\Delta\Angle$).
However, for the changes in node coordinates this is not feasible, since they do not depend linearly on the fluxes due to triangular function relations.
A more appropriate way is to define the node shifts $\Normal\ShiftStep$ and $\Tangential\ShiftStep$ as internal state velocities, so that the required constraints $\vect\RequiredConstraint$ become linear (see \autoref{subsec:required-constraints}).
As they are determined, translation into new coordinates is directly possible using the relations obtained in the previous sections.

The external state variables (such as temperature $\Temperature$ and pressure $\Pressure$) are not considered in the following, since they are considered constant.

\subsection{Dissipation $\Dissipation$}\label{subsec:dissipation}

The dissipation at one node is given by the product node shifts and the respective Gibbs energy derivatives as in \autoref{eq:dissipation-node}.
The Gibbs energy derivatives were determined in previous sections for the different node types.
The dissipation of the whole system is the sum of all node dissipations, since all thermodynamic forces occur at nodes.
Shifting of particles alone does not follow a thermodynamic force, but is determined by the evolution of the related notes.

\begin{equation}
    \Dissipation = \frac{\partial \GibbsEnergy}{\partial \Normal\Shift} \Normal\ShiftStep + \frac{\partial \GibbsEnergy}{\partial \Tangential\Shift} \Tangential\ShiftStep
    \label{eq:dissipation-node}
\end{equation}

\subsection{Dissipation Function $\DissipationFunction$}\label{subsec:dissipation-function}

\subsection{Required Constraints $\vect\RequiredConstraint$}\label{subsec:required-constraints}

\subsection{Additonal Constraints $\vect\AddionalConstraint$}\label{subsec:additonal-constraints}
