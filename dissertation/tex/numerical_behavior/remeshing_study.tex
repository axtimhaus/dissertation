\section{Study on the Influence of Remeshing}

The choice of discretization width is always a trade-off between precision of the numerical procedure and it's computational cost.
Usually one aims at the finest discretization required and the coarsest possible.
Often in simulation tasks, this choice has to be made one time before running the simulation.
In the current case, these needs and possibilities highly vary during progress of simulation.
So in the beginning of the sintering process, where curvatures are high a fine discretization is desireable to not miss the details.
In later stages, curvatures have decreased and fine discretization leads to unnecessarily high effort.

This study aims on evaluating the influence of the remeshing procedures described in \autoref{sec:remeshing-procedures} on the precision of time integration and providing a decent choice of the parameters to use in the following investigations.

\subsection{Construction of the Study}

The study uses a pair of particles of equal size and material, circular in shape with a defined and equal initial distance of contact.
This is a standard case for other sintering models found in literature, so their predictions can be used as reference to the current's ones.

The case matrix of the study has two dimensions.
The first ist the count of nodes per particle in the initial state, which is inverse proportional to the discretization width.
The levels used are \numlist{50;100;200} nodes per particle.
The second is the activation and parameterization of the surface remeshing procedure (\autoref{sec:free-surface-remeshing}).
The levels used are deactivated and the angle thresholds of \qtylist{0.01;0.02;0.05}{\radian}.
As the surface follows a circular shape, high influence of node adding has not to be expected here, so it is disregarded.
The surface distance limit is fixed to the default of \num{3.0}.
As neck remeshing (\autoref{sec:neck-neighborhood-remeshing}) must always be activated to make the simulation work, it is not regarded here distinctly.
Neck remeshing is always used with a limit parameter of \num{0.3} here.

\subsection{Discussion of the Results}
