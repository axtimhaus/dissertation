\section{Study on the Influence of Remeshing}\label{sec:study-surface-remeshing}

The choice of discretization width is always a trade-off between precision of the numerical procedure and it's computational cost.
Usually one aims at the finest discretization required and the coarsest possible.
Often in simulation tasks, this choice has to be made one time before running the simulation.
In the current case, these needs and possibilities highly vary during progress of simulation.
So in the beginning of the sintering process, where curvatures are high a fine discretization is desireable to not miss the details.
In later stages, curvatures have decreased and fine discretization leads to unnecessarily high effort.

This study aims on evaluating the influence of the surface remeshing procedure described in \autoref{sec:remeshing-procedures} on the precision of time integration and providing a decent choice of the parameters to use in the following investigations.

\subsection{Construction of the Study}

The case matrix of the study has two dimensions.
The first is the count of nodes per particle in the initial state, which is inverse proportional to the mean discretization width.
The levels used are \numlist{50;100;200} nodes per particle.
The second is the activation and parameterization of the surface remeshing procedure (\autoref{sec:free-surface-remeshing}).
The levels used are deactivated and the angle thresholds of \qtylist{0.01;0.02;0.05}{\radian}.
As the surface follows a circular shape, high influence of node adding has not to be expected here, so it is disregarded.
The surface distance limit is fixed to the default of \num{3.0}.
Neck remeshing is necessarily activated with a distance limit parameter of \num{0.5}, as was determined as appropriate in \autoref{sec:study-neck-remeshing}.

Targets of investigation are as before the time consumption (duration) of the differently parametrized simulations, the step count needed and the evolution of the step width's during the simulation.
As measure for precision, their agreement in shrinkage and the overall volume loss are regarded.

\subsection{Discussion of the Results}

\begin{figure}
    %     \includegraphics{sim/two_particle/studies/surface_remeshing/step_count}
    \caption{Step Counts and Simulation Time Consumption Used in Dependence on the Maximum Displacement Angle}
    \label{fig:two_particle/studies/surface_remeshing/step_count}
\end{figure}

\autoref{fig:two_particle/studies/surface_remeshing/step_count} shows the count of calculated time steps alongside with the total time consumption for the simulations.

\begin{figure}
    %     \includegraphics{sim/two_particle/studies/surface_remeshing/shrinkage}
    \caption{Shrinkage Result Curves Obtained from the Simulations in Dependence on the Maximum Displacement Angle}
    \label{fig:two_particle/studies/surface_remeshing/shrinkage}
\end{figure}

\autoref{fig:two_particle/studies/surface_remeshing/shrinkage} shows the shrinkage curves obtained from the parameter variation.

\begin{figure}
    %     \includegraphics{sim/two_particle/studies/surface_remeshing/volume_loss}
    \caption{Volume Loss During Simulation in Dependence on the Maximum Displacement Angle}
    \label{fig:two_particle/studies/surface_remeshing/volume_loss}
\end{figure}

\autoref{fig:two_particle/studies/surface_remeshing/volume_loss} shows the loss of volume for both particles (solid and dashed lines) in the simulations with the varied parameter (color).

\begin{figure}
    %     \includegraphics{sim/two_particle/studies/surface_remeshing/time_step_width}
    \caption{Time Step Width During Simulation in Dependence on the Maximum Displacement Angle}
    \label{fig:two_particle/studies/surface_remeshing/time_step_width}
\end{figure}

\autoref{fig:two_particle/studies/surface_remeshing/time_step_width} shows the evolution of the time step width used in the simulations for the distinct cases.

\subsection{Summary}
