\section{Study on the Influence of Neck Remeshing}\label{sec:study-neck-remeshing}

As stated previously, the procedure of remeshing near the neck is absolutely necessary the make the simulation work.
However, there is a free parameter determining at which distance a surface node near the neck may be deleted.
Targeting numerical efficiency it is desireable to have this parameter as large as possible as the time step decreases as already noticed in \autoref{sec:study-step-size}.
Too large limits, however, are expected to decrease the precision especially in regard on geomertical features near the neck.
In the previous studies the limit was arbitrarily chosen as \num{0.5}.
In this study the parameter will be altered to get an idea of it's effects.

\subsection{Construction of the Study}

The case matrix of the study has two dimensions.
The first is the count of nodes per particle in the initial state, which is inverse proportional to the discretization width.
The levels used are \numlist{50;100;200} nodes per particle.
The second is the parameterization of the neck remeshing procedure (\autoref{sec:neck-neighborhood-remeshing}).
The levels of the distance limit parameter used are \qtylist{0.3;0.5;0.7}{\radian}.
Surface remeshing is deactivated as before.

Targets of investigation are as before the time consumption (duration) of the differently parametrized simulations, the step count needed and the evolution of the step width's during the simulation.
As measure for precision, their agreement in shrinkage and the overall volume loss are regarded.

\subsection{Discussion of the Results}

\begin{figure}
    %     \includegraphics{sim/two_particle/studies/neck_remeshing/step_count}
    \caption{Step Counts and Simulation Time Consumption Used in Dependence on the Maximum Displacement Angle}
    \label{fig:two_particle/studies/neck_remeshing/step_count}
\end{figure}

\autoref{fig:two_particle/studies/neck_remeshing/step_count} shows the count of calculated time steps alongside with the total time consumption for the simulations.

\begin{figure}
    %     \includegraphics{sim/two_particle/studies/neck_remeshing/shrinkage}
    \caption{Shrinkage Result Curves Obtained from the Simulations in Dependence on the Maximum Displacement Angle}
    \label{fig:two_particle/studies/neck_remeshing/shrinkage}
\end{figure}

\autoref{fig:two_particle/studies/neck_remeshing/shrinkage} shows the shrinkage curves obtained from the parameter variation.

\begin{figure}
    %     \includegraphics{sim/two_particle/studies/neck_remeshing/plots/volume_loss}
    \caption{Volume Loss During Simulation in Dependence on the Maximum Displacement Angle}
    \label{fig:two_particle/studies/neck_remeshing/volume_loss}
\end{figure}

\autoref{fig:two_particle/studies/neck_remeshing/volume_loss} shows the loss of volume for both particles (solid and dashed lines) in the simulations with the varied parameter (color).

\begin{figure}
    %     \includegraphics{sim/two_particle/studies/neck_remeshing/time_step_width}
    \caption{Time Step Width During Simulation in Dependence on the Maximum Displacement Angle}
    \label{fig:two_particle/studies/neck_remeshing/time_step_width}
\end{figure}

\autoref{fig:two_particle/studies/neck_remeshing/time_step_width} shows the evolution of the time step width used in the simulations for the distinct cases.

\subsection{Summary}
