\section{Statistical Powder Modeling} \label{sec:statistical-powder-modeling}

As every particle is unique, the only way to characterize a powder is by statistical quantities, either scalar properties like the mean or by a distribution.
\autoref{sec:powder-characterization} discussed the usual methods of characterizing powders.
In this chapter, a statistical approach directly based on the distribution of size, shape and substance parameters shall be developed to generate inputs for the simulation.
The approach resides on random sampling of particles obeying these distributions.

\section{Modeling of Particle Shape Properties by a Shape Function}\label{sec:particle-shape-function}

\begin{figure}
    \begin{subfigure}{0.5\linewidth}
        \includegraphics[width=\linewidth]{img/model_development/particle_shape_function_o}
        \caption{Ovality $o$}
        \label{fig:model_development/particle_shape_function_o}
    \end{subfigure}%
    \begin{subfigure}{0.5\linewidth}
        \includegraphics[width=\linewidth]{img/model_development/particle_shape_function_n}
        \caption{Peak Count $n$}
        \label{fig:model_development/particle_shape_function_n}
    \end{subfigure}
    \begin{subfigure}{0.5\linewidth}
        \includegraphics[width=\linewidth]{img/model_development/particle_shape_function_h}
        \caption{Peak Height $h$}
        \label{fig:model_development/particle_shape_function_h}
    \end{subfigure}%
    \begin{subfigure}{0.5\linewidth}
        \includegraphics[width=\linewidth]{img/model_development/particle_shape_function_p}
        \caption{Phase Shift $p$}
        \label{fig:model_development/particle_shape_function_p}
    \end{subfigure}
    \caption{Influence of the Shape Function Parameters on the Resulting Shape}
    \label{fig:model_development/particle_shape_function}
\end{figure}

Common representation of particles is done using circles (2D) or spheres (3D).
However, these neglect non-ideal shape features which can be of great importance.
As each particle is unique, direct representation by an actual contour is not practical.
The current approach will use a shape function which includes ovality and first-order waves.
Multiple variants of this approach with varying depth of detail are imaginable and can be used instead of the one derived here.

The shape function will follow the general form
\begin{equation}
    \Radius\left( \Angle \right) = \Radius_0 \cdot \left[ \prod_i \ShapeFunction_i\left( \Angle \right) \right]
    \label{eq:shape-function-general-form}
\end{equation}
with $\Radius$ as the local particle radius, $\Angle$ as the angle coordinate and the $f_i$ as additional terms adding a specific shape feature.
If there aren't any $f_i$, the model will fall back to the usual circular shape with constant radius $\Radius_0$ as the empty product has the value 1.
The $f_i$ will generally include some unknown parameters which can be used to characterize the particle shape later on.
If the $f_i$ parameters have a notion of neutrality or deactivation, the value of $f_i$ must be 1 in this cases.

For the regard of ovality, we define a parameter $\Ovality \in [ 1, \infty ] $, which shall follow the common sense that
\begin{equation}
    \Ovality = \frac{a}{b}
    \label{eq:ovality-o}
\end{equation}
when $a$ and $b$ are the largest respectively the smallest measure of the particle.
We can also understand $a$ and $b$ as the largest and smallest radii of an ellipse.
In relation to the base radius $\Radius_0$ we can calculate them using
\begin{align}
    a &= \Radius_0 \cdot \sqrt{\Ovality} \\
    b &= \Radius_0 / \sqrt{\Ovality}
    \label{eq:ellipse-a-b-by-ovality}
\end{align}
for which \autoref{eq:ovality-o} holds.
From the polar form of the ellipse equation
\begin{equation}
    \Radius \left( \Angle \right) = \frac{ab}{\sqrt{\left( a \sin \Angle \right)^2 + \left( b \cos \Angle \right)^2  }}
    \label{eq:ellipse-polar}
\end{equation}
we obtain, when inserting \autoref{eq:ellipse-a-b-by-ovality}, the equation for the shape function of ovality $f_{\Ovality}$ as
\begin{equation}
    \ShapeFunction_{\Ovality} = \sqrt{\frac{\Ovality}{\Ovality^2 \sin^2 \Angle + \cos^2 \Angle}}
    \label{eq:f-ovality}
\end{equation}
\autoref{fig:model_development/particle_shape_function_o} displays the influence of the ovality parameter on the resulting shape.

First order waves shall be modeled using a cosine function.
With a relative height of the peaks $\WaveHeight_1 \in ]0, 1]$, a count of peaks $\WaveCount_1 \in [1 \To]$ and a relative phase shift $\WaveShift_1 \in [0, 0.5[$ we define
\begin{equation}
    \ShapeFunction_1 = \WaveHeight_1 \cdot \cos \left( \WaveCount_1\Angle + 2\PI \WaveShift_1\right) + 1
    \label{eq:f-first-order-waves}
\end{equation}
As can be seen in \autoref{fig:model_development/particle_shape_function_n} and \autoref{fig:model_development/particle_shape_function_h},
the only way to deactivate this feature is to set $\WaveHeight_1=0$, since $\WaveCount_1=0$ is not allowed as it would introduce an ambiguity by adding a constant contribution to the radius depending on $\WaveHeight_1$ and $\WaveShift_1$.
Note especially the influence of the phase shift in \autoref{fig:model_development/particle_shape_function_p}.
The phase shift enables moving the waves relative to the ovality but has a major numerical flaw.
Due to it's nature as an argument to the cosine function, it's effect is periodic with a period of \num{1}.
If also bringing particle rotation in consideration, also values $\WaveShift_1 \ge \num{0.5}$ are ambiguous, as the same effect can be achieved by using $\WaveShift_1 - \num{0.5}$ and rotating the whole particle by \ang{180}.
So the allowed value range given above must be ensured to avoid ambiguity.

\section{Modeling of Powders and Powder Mixtures}\label{sec:powders-powder-mixtures}

A single powder fraction can be modelled using the shape function given above by defining the statistical distribution of the base radius $\Radius_0$, and the respective shape function parameters.
This implicitly assumes, that the shape features follow the same characteristic in all particle sizes.
The particle size (represented by $\Radius_0$) often follows complicated distributions in real world scenarios, sometimes multi-modal.
So the most practical approach is to use the empirical distribution here.
As the shape parameters have defined value ranges, it is feasible to apply statistical distributions that support these out of the box.
The ovality $\Ovality$ has a lower limit of \num{1} and no upper limit with the highest probability somewhere near \num{1}, since needle-like particles are rare due to break-up.
So a shifted Gamma-distribution is feasible here, as it supports a variety of shapes with only a lower limit.
The wave height $\WaveHeight$ and wave shift $\WaveShift$ both have lower and upper limits, which must be ensured.
Here, a Beta-distribution seems applicable.
The wave count $\WaveCount$ is of categorical type.
Given these distributions a sample of particles, following these statistics, can be generated by sampling from these distributions as discussed in \autoref{sec:monte-carlo}.

Usually, powder mixtures applied in real sintering scenarios do not consist of only one powder fraction.
To obtain optimal properties, multiple fractions of certain particle size distributions are mixed together.
Often they originate from different processes and therefore have different shape characteristics.
They may also consist of different substances.
This mixing process is here modeled by categorical distributions in a second layer to the procedure described above for simple powders.
Usually, practical powder mixtures are given by volume fractions.
The procedure here, however, needs particle count fractions, as in each sampling process a fixed number of particles is generated.

Conversion between volume fraction $\MixtureFraction_{\Volume}$ and count fraction $\MixtureFraction_{\ParticleCount}$ of a powder fraction $i$ is done using the equivalent volume of a particle, which is only dependent on $\Radius_0$, since the shape functions above preserve the volume.
\begin{equation}
    \MixtureFraction_{\ParticleCount i} = \frac{\MixtureFraction_{\Volume i}}{\PI \Radius_{0 i} ^ 2}
    \label{eq:equivalent-particle-volume}
\end{equation}
This assumes that all powder fractions have the same packing density.
The influence of varying particle shape on the packing density is neglected here, as it's estimation from theory is complicated, an unsolved problem out of the scope of this work.
