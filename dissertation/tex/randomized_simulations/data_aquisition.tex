\section{Aquisition of Particle Shape Data}\label{sec:randomized/data_aquisition}

The underlying data is the same as in \textcite{Weiner2022a}, but there a different shape function has been used.
The fits are therefore repeated with the current shape function as discussed in \cref{sec:statistical-powder-modeling}.
The dataset (available in \cite{Weiner2025ThesisSupp}) includes contours of aggregate composite particles produced by sintering, crushing and sieving as decribed in \textcite{Zienert2022, Zienert2022a}, which have been extracted from laser scanning micrographs.

The fit is accomplished in all cases by least squares optimization on the euclidian distance between data points and the model function predictions.
The experimental particle contour is shifted prior to fitting, so that its barycenter is in $(0, 0)$.

\subsection{Circular Particles}

\begin{figure}
    \includegraphics{data/morphology/batches/hist_circular/all}
    \caption{Particle Size Distribution when Assuming Circular Particles}
    \label{fig:morphology/batches/hist_circular/all}
\end{figure}

In the first case, all shape terms are deativated in the function and fitting is solely done assuming circular particle shape.
So, the only parameter regarded is the particle radius $\Radius_0$.
\Cref{fig:morphology/batches/hist_circular/all} shows the histogram of the fit results alongside the theoretical distribution fit.
Note the bimodal distribution shape of the histogram.

For particle size distributions it is common to use a Weibull distribution (\cref{eq:weibull}), with the shape parameter $k$, the scale parameter $S$ and the location parameter $L$.
The support of the distribution is $x \in [L, \infty)$.
As particle sizes must always be $> 0$, the location $L=0$ is used here.
\begin{equation}
    \Probability(x) = \frac{k}{S} \left(\frac{x - L}{S}\right)^{k-1} \exp \left[ -\left(\frac{x - L}{T}\right)^k \right]
    \label{eq:weibull}
\end{equation}
The expectation $\Expectation$ of the Weibull distribution is given by \cref{eq:weibull-expectation} with the gamma function $\GammaFunction$.
The expected radius $\Expectation{\Radius_0}$ will be used as the reference radius to calculate the characteristic time norm according to \cref{eq:normalized_time}.
\begin{equation}
    \Expectation = S \cdot \GammaFunction \left( 1 + \frac1k \right)
    \label{eq:weibull-expectation}
\end{equation}

To obtain a bimodal distribution, a mixture distribution of two Weibull distributions is used with parameters $(k_1, S_1)$ and $(k_2, S_2)$.
The probability of sampling from the first one is the weight $w$, whereas of choosing the second $1-w$, respectively.

\subsection{Oval Particles}

\begin{figure}
    \includegraphics{data/morphology/batches/hist_oval/all}
    \caption{Particle Size adn Ovality Distributions when Assuming Oval Particles}
    \label{fig:morphology/batches/hist_oval/all}
\end{figure}

In the second case, the ovality term (\cref{eq:f-ovality}) is added to the shape function.
Beneath the ovality $\Ovality$, also the rotation $\RotationAngle$ of the particle has to be fitted, since the shape function is no longer invariate counter rotation.
As the latter is needed for correct fitting but represents no actual particle shape feature but the placement of the particle, it is not analysed futher in the following.
Fitting the rotation is numerically unstable as there are many local extrema.
To get all local extrema a brute force method is applied which chooses a couple of linearly spaced initial values for the rotaion parameter and selects the best achieved fit afterwards.
This method is simple and easy to implement while the impact of computational effort hardly matters here.
\Cref{fig:morphology/batches/hist_oval/all} shows the histograms of the fit results alongside the theoretical distribution fits.
The ovality $\Ovality$ is modelled by a single Weibull distribution.
In contrast to particle size, here a location $L=1$ is used, as ovality $\Ovality \ge 1$.

\subsection{First-Order-Waved Particles}

\begin{figure}
    \includegraphics{data/morphology/batches/hist_shape/all}
    \caption{Particle Size adn Ovality Distributions when Assuming Oval Particles with First Order Waves}
    \label{fig:morphology/batches/hist_shape/all}
\end{figure}

In the third case, first-order waves according to \cref{eq:f-first-order-waves} are considered, too.
So, the fit additioally includes the parameters wave height $\WaveHeight$, wave count $\WaveCount$ and wave shift $\WaveShift$.
As before the rotation has to be fitted accordingly.
As $\WaveCount$ is integer valued and only few distinct values have to be expected, it is treated similarly to the rotation by a brute force method.
Mixed continuous and integer optimization is a generally hard problem, were several sophisticated algorithms exist.
For the current case they have no computational benefit, as they have typically high effort per step and the value range of $\WaveCount$ is small.
\Cref{fig:morphology/batches/hist_shape/all} shows the histograms of the fit results alongside the theoretical distribution fits.

The wave height $\WaveHeight$ is modeled using a beta distribution, which supports a variety of shapes and has a defined support for $[0, 1]$ which fits ideally to the parameter.
$\alpha$ and $\beta$ are the free parameters of the distribution.
\begin{equation}
    \Probability(\WaveHeight) = \frac{1}{\Beta(\alpha, \beta)} \WaveHeight^{\alpha - 1} \left( 1 - \WaveHeight \right) ^ {\beta - 1}
    \label{eq:beta-wave-height}
\end{equation}

The wave count parameter $\WaveCount$ is of discrete nature and only few distinct values were present in the data.
So, the parameter is not modelled by a theoretical distribution but handeled as a categorical variable with defined relative frequencies.
The wave shift $\WaveShift$ is suprisingly close to a uniform distribution, so no fitting of an distribution function is required here.

\subsection{Comparison of the Three Approaches}

%1694-48 as example

\begin{figure}
    \centering
    \begin{subfigure}{0.5\linewidth}
        \centering
        \includegraphics[scale=0.5]{img/randomized/1694_48_circular}
        \caption{Circular}
        \label{fig:comparison-particle-shape-fits-circular}
    \end{subfigure}%
    \begin{subfigure}{0.5\linewidth}
        \centering
        \includegraphics[scale=0.5]{img/randomized/1694_48_oval}
        \caption{Oval}
        \label{fig:comparison-particle-shape-fits-oval}
    \end{subfigure}
    \begin{subfigure}{0.5\linewidth}
        \centering
        \includegraphics[scale=0.5]{img/randomized/1694_48_shape}
        \caption{First-Order Waves}
        \label{fig:comparison-particle-shape-fits-shape}
    \end{subfigure}%
    \caption{Comparison of the Three Shape Fit Approaches for a Typical Particle}
    \label{fig:comparison-particle-shape-fits}
\end{figure}

The obtained particle size distributions from the three aproaches are quite similar, despite the changes in the shape descriptions.
Due to improved shape accuracy, the mean of particle size $\Radius_0$ is slightly decreasing from \qtyrange{1386}{1357}{\micro\meter}.
To ensure comparability, the reference radius for normalization will be fixed to the approximate mean value of \qty{1370}{\micro\meter} for the rest of this chapter.
This radius ist also used as the radius of the nominal case particles.

\Cref{fig:comparison-particle-shape-fits} shows a typical particle from the dataset, which has been fitted by the three distinct approaches.
One could clearly observe the improved shape representation when adding the additional terms.
Fine surface structures obviously can not be represented by the models, but the overall shape characteristic is well modeled by the complete shape function with ovality and first-order waves.
