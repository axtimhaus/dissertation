\section{Construction of the Study}\label{sec:randomized/construction}

The study will stick to three-particle arrangements to avoid major difficulties in construction of the initial state.
As the size and shape of the particles will now be variable, the initial position can not be given a-priori but must be determined numerically.
The study will compare three cases: three powders with equal particle size distribution but one assumed to have circular particles,
one where only an ovality according to \cref{eq:f-ovality} is considered
and one where also first order waves according to \cref{eq:f-first-order-waves} are considered.
The differences and similarities of the three cases will be investigated to rate the need for inclusion of this data into sintering simulations.
The core question is if the first order waves have major influence or if regard on ovality is enough, since ovality is a parameter that can be measured without problems using state of the art automatic particle classifiers.
First order waves, however, require an high amount of manual work.
The author has already published a paper describing a method to obtain these for aggregate particles via laser scanning microscopy images and leat-square fitting in \textcite{Weiner2022a}, but there another approach of sintering model is used.
The data obtained there will be reused in this study.

\subsection{Statistical Description of the Example Powder}

For each of the shape parameters included in the chosen shape function, a suitable distribution function has to be selected and fitted on the experimentally obtained data.
The types of the distributions are selected according to their typical shape and support.
The choice and fitting will be discussed in \cref{sec:randomized/data_aquisition}.

\subsection{Creation of the Initial State}

The intial state generation starts by sampling three particles randomly.
For each a radius $\Radius_0$ and one of every shape parameter is sampled from the given distribution.
Also, the rotation of each particle is sampled from a uniform distribution with value range $\RotationAngle \in [0, 2\PI)$.
The initial positions of the particles are created in sufficient distance to each other, so that it is impossible for them to be in contact.

The first particle is kept fixed in space.
Then, the second is moved along the vector towards the center of the first in small steps until contact is created by a defined minimum intersection of the two surfaces.
Then the third is moved along the sum of of the vectors from the thirds center to the others' centers.
If contact with one is created, then the vector towards this particle is counted negative to allow sideward movement.
When the third particle has contact to both, the procedure is completed.
