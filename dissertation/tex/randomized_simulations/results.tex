\section{Simulation Results}\label{sec:randomized/results}

The \crefrange{fig:randomized/cases/circular/shrinkage}{fig:randomized/cases/shape/neck_size} show the shrinkage respectively neck size evolution of the randomized simulations.
The distinct samples are drawn as a partially transparent lines, so that the density of samples can be estimated visually by the intensity of the color.
For comparison the arithmetic mean of the samples and the result of the nominal case are drawn alongside.
The arithmetic mean curve is calculated by interpolating the sample curves at the respective time.
When the respective sample curve was does not provide a value for the respective time, because the simulation was already canceled due to pore closing, than the last value of the curve is taken.
This is equivalent to the assumption that the particle group is inert (does not evolve any further) after pore closing has happened.
This assumption has to be imposed, as simulation after pore closing is not supported in the current state of the model (see \cref{sec:contact-study}).
Extrapolation of the curves is not feasible, as they approach a singularity (approach infinite slope) when the pore is almost closed.
In each figure, also two histograms at the normalized times $\Time/\TimeNorm_{\Surface} = \num[print-unity-mantissa=false]{1e-6}$ and $\Time/\TimeNorm_{\Surface} = \num[print-unity-mantissa=false]{1e-6}$ are drawn, which also show the respective mean value and the standard deviation.
The histograms are calculated using the same assumption on pore closing.

To measure the size of an effect, the absolute effect~$\EffectSize$ (difference of means) is often not sufficient, as it neglects the spread of the data within each group.
\textcite{Cohen1988} defined a relative effect size~$\RelativeEffectSize$, also known after the author as Cohen's effect size.
It is given as the absolute effect size $\EffectSize$ divided by the standard deviation $\StandardDeviation$ as in \cref{eq:relative-effect-size}.
Which standard deviation to take is an issue, especially when sample sizes are different.
For equal sample sizes \textcite{Cohen1988} gave the expression in \cref{eq:combined-standard-deviation}.
Here, the case is simpler, as the comparison is done between each randomized and the nominal case, which is fully certain.
So, the reference standard deviation is just that of the randomized case.
\Cref{tab:randomized-effects} lists the obtained means and standard deviation for the respective cases alongside with the absolute and relative effect sizes.
For rating the value of Cohen's effect size, \textcite{Sawilowsky2009} gave rules of thumb, according to which an effect of $\RelativeEffectSize \approx \num{0.2}$ is to be classified as small, $\RelativeEffectSize \approx \num{0.5}$ as medium, $\RelativeEffectSize \approx \num{0.8}$ as large and  $\RelativeEffectSize \approx \num{1.2}$ as very large.
\begin{equation}
    \RelativeEffectSize = \frac{\EffectSize}{\StandardDeviation} = \frac{\Estimated\Expectation_1 - \Estimated\Expectation_2}{\StandardDeviation}
    \label{eq:relative-effect-size}
\end{equation}
\begin{equation}
    \bar\StandardDeviation = \sqrt{\frac12 \left( \StandardDeviation_1^2 + \StandardDeviation_2^2 \right) }
    \label{eq:combined-standard-deviation}
\end{equation}

Below, the results of shrinkage and neck size will be described with respect to the plots and the calculated effect sizes under consideration of the effect size rules of thumb.

\begin{table}
    \caption{Comparison of Means and Standard Deviations and their Effect Sizes for the Distinct Cases of Randomized Simulations}
    \label{tab:randomized-effects}
    \begin{tblr}{
            colspec={lXXXXXXXX},
            column{2-9} = {r}
        }
        \toprule
        & \SetCell[c=2]{c} Nominal & & \SetCell[c=2]{c} Circular & & \SetCell[c=2]{c} Oval & & \SetCell[c=2]{c} Waves & \\ \cmidrule[lr]{2-3} \cmidrule[lr]{4-5} \cmidrule[lr]{6-7} \cmidrule[lr]{8-9}
        $\Time / \TimeNorm_{\Surface}$ & \num[print-unity-mantissa=false]{1e-6} & \num[print-unity-mantissa=false]{1e-4} & \num[print-unity-mantissa=false]{1e-6} & \num[print-unity-mantissa=false]{1e-4} & \num[print-unity-mantissa=false]{1e-6} & \num[print-unity-mantissa=false]{1e-4} & \num[print-unity-mantissa=false]{1e-6} & \num[print-unity-mantissa=false]{1e-4} \\
        \midrule
        Shrinkage \\ \cmidrule[lr]{1}
        $\Estimated\Expectation$ in \unit{\percent} & \num{1.189} & \num{5.502} & \num{1.156} & \num{4.843} & \num{1.139} & \num{4.778} & \num{1.185} & \num{4.377} \\
        $\Estimated\StandardDeviation$ in \unit{\percent} & -- & -- & \num{0.428} & \num{1.038} & \num{0.364} & \num{1.063} & \num{0.468} & \num{1.559} \\
        $\EffectSize$ in \unit{\percent} & -- & -- & \num{-0.033} & \num{-0.659} & \num{-0.050} & \num{-0.724} & \num{-0.004} & \num{-1.125} \\
        $\RelativeEffectSize$ in \unit{\percent} & -- & -- & \num{-0.077} & \num{-0.635} & \num{-0.137} & \num{-0.681} & \num{-0.009} & \num{-0.722} \\
        Neck Size \\ \cmidrule[lr]{1}
        $\Estimated\Expectation$ in \unit{\percent} & \num{24.861} & \num{45.734} & \num{23.664} & \num{41.792} & \num{23.267} & \num{41.782} & \num{21.502} & \num{38.138} \\
        $\Estimated\StandardDeviation$ in \unit{\percent} & -- & -- & \num{3.856}  & \num{8.670} & \num{3.215} & \num{7.427} & \num{4.017} & \num{9.369} \\
        $\EffectSize$ in \unit{\percent} & -- & -- & \num{-1.197} & \num{-3.942} & \num{-1.594} & \num{-3.952} & \num{-3.359} & \num{-7.596} \\
        $\RelativeEffectSize$ in \unit{\percent} & -- & -- & \num{-0.310} & \num{-0.458} & \num{-0.496} & \num{-0.532} & \num{-0.836} & \num{-0.811} \\
        \bottomrule
    \end{tblr}
\end{table}

\subsection{Observations on Shrinkage}

\begin{figure}
    \centering
    \includegraphics[width=\linewidth]{sim/randomized/cases/circular/shrinkage}
    \caption{Shrinkage of the Circular Case}
    \label{fig:randomized/cases/circular/shrinkage}
\end{figure}

\begin{figure}
    \centering
    \includegraphics[width=\linewidth]{sim/randomized/cases/oval/shrinkage}
    \caption{Shrinkage of the Oval Case}
    \label{fig:randomized/cases/oval/shrinkage}
\end{figure}

\begin{figure}
    \centering
    \includegraphics[width=\linewidth]{sim/randomized/cases/shape/shrinkage}
    \caption{Shrinkage of the First-Order Waves Case}
    \label{fig:randomized/cases/shape/shrinkage}
\end{figure}

The \crefrange{fig:randomized/cases/circular/shrinkage}{fig:randomized/cases/shape/shrinkage} show the shrinkage results of the randomized simulations.

Regarding the early stages, the following observations are to be noted:
\begin{enumerate}
    \item The deviations between the nominal and the randomized cases is very small, and both curves are visually in well agreement.
    \item The relative effect sizes can be classified as small or less.
        The effect size of the waved case is even much lower than of the other ones, which is related to the fact that in this case the time section at $\Time/\TimeNorm_{\Surface} = \num[print-unity-mantissa=false]{1e-6}$ is accidentally near an intersection of the mean and nominal curves.
    \item The standard deviation of the oval case is slightly smaller compared to the circular case, whereas that of the waved case is slightly higher.
    \item The histograms show an approximately symmetric shape when plotted in logarithmic scale.
\end{enumerate}

Regarding the late stages, the following observations are to be noted:
\begin{enumerate}
    \item The deviations between the nominal and the randomized cases is much larger than in early stages.
    \item The relative effect size has to be classified between medium and large, where the effect slightly increases from circular, over oval, to waved.
    \item The circular and oval case have comparable standard deviation, whereas that of the waved case is remarkably higher.
    \item The waved case has a drastically higher absolute effect, but the relative effect, although higher than that of the oval case, is not that pronounced due to the higher standard deviation.
    \item  The histograms show a heavy tail on the left side, formed by samples with early pore closing.
        The heavy tail is especially pronounced in the waved case.
        This is in accordance with the smoother transition into the stationary state.
\end{enumerate}

Regarding the overall trends with time, the following observations are to be noted:
\begin{enumerate}
    \item The nominal curve shows the typical behavior of a closing pore with increasing slope as already observed in \cref{sec:contact-study} and in the individual samples.
    \item The mean curve shows a flattening characteristic with transition into a stationary state.
        The transition into the stationary state is more sudden in the circular case over the oval to the waved case.
    \item In all cases the standard deviation increases with time.
\end{enumerate}

\subsection{Observations on Neck Size}

\begin{figure}
    \centering
    \includegraphics[width=\linewidth]{sim/randomized/cases/circular/neck_size}
    \caption{Neck Size of the Circular Case}
    \label{fig:randomized/cases/circular/neck_size}
\end{figure}

\begin{figure}
    \centering
    \includegraphics[width=\linewidth]{sim/randomized/cases/oval/neck_size}
    \caption{Neck Size of the Oval Case}
    \label{fig:randomized/cases/oval/neck_size}
\end{figure}

\begin{figure}
    \centering
    \includegraphics[width=\linewidth]{sim/randomized/cases/shape/neck_size}
    \caption{Neck Size of the First-Order Waves Case}
    \label{fig:randomized/cases/shape/neck_size}
\end{figure}

The \crefrange{fig:randomized/cases/circular/neck_size}{fig:randomized/cases/shape/neck_size} show the neck size results of the randomized simulations.

Regarding the early stages, the following observations are to be noted:
\begin{enumerate}
    \item In contrast to shrinkage, neck size is also in early stages significantly affected by the randomized consideration of size and shape features.
    \item The absolute effect heavily increases from the circular, over the oval to the waved case.
    \item The standard deviation of the oval case is smaller compared to the circular case, whereas this of the waved case is higher.
    \item The relative effect sizes are to be classified as small to medium for the circular, medium for the oval case and large for the waved case.between
    \item The histogram of the circular case (\cref{fig:randomized/cases/circular/neck_size}) has a heavily right skewed shape when plotted in logarithmic scale, which corresponds to a symmetric shape in linear scale.
        This skew vanishes over the oval case to the waved case.
\end{enumerate}

Regarding the late stages, the following observations are to be noted:
\begin{enumerate}
    \item The absolute effect is similar for the circular and the oval case.
        The waved case has a remarkably higher effect.
    \item The standard deviation follows the same trend as in early stages.
    \item The relative effect sizes are similar for the circular and oval cases, both to be classified as medium.
        The waved case has a higher effect, to be classified as large.
    \item The histograms show a heavily right skewed shape when plotted in logarithmic scale.
        The tail gets heavier from the circular, over the oval, to the waved case.
\end{enumerate}

Regarding the overall trends with time, the following observations are to be noted:
\begin{enumerate}
    \item Similarly to shrinkage, the nominal curve shows the typical pore closing behavior, whereas the mean curve transits into a stationary state with the same trends observed before.
    \item As in shrinkage, the standard deviation increases with time.
    \item In contrast to the shrinkage, the relative effect sizes are not remarkably increasing with time, the larger absolute effect is canceled out by the increase in standard deviation.
\end{enumerate}
