\chapter{Investigation of the Model's Numerical Behavior}\label{ch:numerical-behavior}

The studies performed in this chapter aim at the investigation of the model's numerical behavior before transition to investigations on it's physical implications.
The test case regarded is identical in all of these.
A pair of circular particles of equal size and equal material properties is used.
The material data applied is that of copper (see \autoref{ch:material-data}), however, the exact values of these are not important, as all data is displayed and analysed in dimensionless manner.
The most important point not covered by the normation is the ratio of grain boundary and surface energy, which is taken as $\frac{\InterfaceEnergy_{\GrainBoundary}}{\InterfaceEnergy_{\Surface}}=\num{0.5}$, as a difference in those is necessary to achieve a main driving force.
The diffusion coefficents on both types of interfaces are taken as equal.

This coice was made as the case of two equal particles is the standard case of all microscopic sintering models.
Additonally, during development and testing it has been found that circular particles are more prone to numerical instabilities due to their smooth surface, which is in or near the equilibrium in regions far from the neck.
This implies that there are always regions of large and of negelctable driving force present in the current state.
Regions near the local equilibrium tend to show fluctuations with alternating gradient signs and perhaps instabilities, as due to the local formulation this is the only way to represent long-distance effects in the model.

\section{Study on the Influence of the Time Step Size}\label{sec:study-step-size}

In \autoref{subsec:time-integration-scheme} and \autoref{eq:maximum-angle-difference} the limit of the time step width to use during time integration was given.
As currently no theoretical limit for stability can be given, the free parameter has to be chosen empirically.
This study aims at investigating the numerical behavior of the simulation with different maximum displacement angles per time step and fix one for the remains of this work.

\subsection{Construction of the Study}

The maximum displacement angle is varied between \qtyrange{0.001}{0.1}{\radian}.
Surface remeshing is deactivated to keep the finely discretized surface and avoid superposition of the effects.
Neck remeshing is necessarily activated with a distance limit parameter of \num{0.5}, arbitrarily chosen.

Targets of investigation are the time consumption (duration) of the differently parametrized simulations, the step count needed and the evolution of the step width's during the simulation.
As measure for precision, their agreement in shrinkage and the overall volume loss are regarded.

\subsection{Discussion of the Results}

\begin{figure}
    %     \includegraphics{sim/two_particle/studies/time_step/step_count}
    \caption{Step Counts and Simulation Time Consumption Used in Dependence on the Maximum Displacement Angle}
    \label{fig:two_particle/studies/time_step/step_count}
\end{figure}

\autoref{fig:two_particle/studies/time_step/step_count} shows the count of calculated time steps alongside with the total time consumption for the simulations.
The most striking point is that the step count does not directly correlate with the duration.
The step count shows a minimum around \qty{0.02}{\radian}, whereas the time consumption is constantly low till \qty{0.02}{\radian} and then rises rapidly.
This is explained through the superimposed influence of the time needed to calculate one time step.
With small time steps, the solution of the current time step is close to the one of the previous, which is used as initial guess in the Newton procedure.
Thus, the routine needs fewer iterations to determine the solution.
With large time steps, however, the new solution is more distant, especially when fluctuations of even surfaces occur in later stages.
This means slower convergence of equation solving.
Moreover, from the calculation logs can be observed that the convergence to the trivial solution as described in \autoref{subsec:step-system-solution} more frequently occurs when the parameter takes higher values.

\begin{figure}
    %     \includegraphics{sim/two_particle/studies/time_step/shrinkage}
    \caption{Shrinkage Result Curves Obtained from the Simulations in Dependence on the Maximum Displacement Angle}
    \label{fig:two_particle/studies/time_step/shrinkage}
\end{figure}

\autoref{fig:two_particle/studies/time_step/shrinkage} shows the shrinkage curves obtained from the parameter variation.
Except for the highest \qty{0.1}{\radian} they are all in well agreement.
This indicates, that the latter is definitely to high to obtain accurate results.
In the initial stage, the differences are larger than in the final stages, since the conditions change faster their and so a smaller step width could be desireable.
However, these differences are equalized so that the results are in well agreement in the later stages nevertheless.

\begin{figure}
    %     \includegraphics{sim/two_particle/studies/time_step/volume_loss}
    \caption{Volume Loss During Simulation in Dependence on the Maximum Displacement Angle}
    \label{fig:two_particle/studies/time_step/volume_loss}
\end{figure}

\autoref{fig:two_particle/studies/time_step/volume_loss} shows the loss of volume for both particles (solid and dashed lines) in the simulations with the varied parameter (color).
For the values below \qty{0.01}{\radian} the major source of error are the remeshing steps (jumps in the shown curves), between which the error does not alter very much.
For the higher values the error originating in the actual time integration is much more pronounced.
Especially for \qty{0.05} and above the error escalates in the later stages.
This can be explained by the occurrence of significant fluctuations on the particle surface (numerical instabilities).

\begin{figure}
    %     \includegraphics{sim/two_particle/studies/time_step/time_step_width}
    \caption{Time Step Width During Simulation in Dependence on the Maximum Displacement Angle}
    \label{fig:two_particle/studies/time_step/time_step_width}
\end{figure}

\autoref{fig:two_particle/studies/time_step/time_step_width} shows the evolution of the time step width used in the simulations for the distinct cases.
It is conspicuous that simulations with the parameter values that were noticed as instable previously, also a high variance in time step width is present.
For lower values the time step width follows the general trend that after neck remeshing has deleted a surface node, the step jumps up and than decreases quite continuously till the next remshing as the neck node approaches the next surface node.
This indicates, that an early neck remshing would be desireable for efficient computation.
This issue will be adressed in detail in \autoref{sec:study-neck-remeshing}.

\subsection{Summary}

The discussed results show, that lower values of the maximum displacement angle are generally preferable.
Higher values lead as expected to numerical errors, however do not improve the time consumption of the solution due to convergence issues of the Newton procedure.
Small time steps allow faster solution due to fewer iterations per step and provide better precision in general.
With regard to the high memory usage for result data storing the maximum displacement angle \qty{0.005}{\radian} is chosen for the following investigation as it provides the best trade-off.
Lower values did not provide better performance in the test case but produce significantly higher amounts of data.

\section{Study on the Influence of Neck Remeshing}\label{sec:study-neck-remeshing}

As stated previously, the procedure of remeshing near the neck is absolutely necessary to make the simulation work.
However, there is a free parameter determining at which distance a surface node near the neck may be deleted.
Targeting numerical efficiency it is desirable to have this parameter as large as possible as the time step decreases as already noticed in \autoref{sec:study-step-size}.
Too large limits, however, are expected to decrease the precision especially in regard on geometrical features near the neck.
In the previous studies the limit was arbitrarily chosen as \num{0.5}.
In this study the parameter will be altered to get an idea of it's effects.

\subsection{Construction of the Study}

The case matrix of the study has two dimensions.
The first is the count of nodes per particle in the initial state, which is inverse proportional to the discretization width.
The levels used are \numlist{50;100;200} nodes per particle.
The second is the parameterization of the neck remeshing procedure (\autoref{sec:neck-neighborhood-remeshing}).
The levels of the distance limit parameter used are \numlist{0.3;0.5;0.7}.
Surface remeshing is deactivated as before.

Targets of investigation are as before the time consumption (duration) of the simulations, the step count needed and the evolution of the step width's during the simulation.
As measure for precision, their agreement in shrinkage and the overall volume loss are regarded.

\subsection{Discussion of the Results}

\begin{figure}
    \begin{subfigure}{0.5\linewidth}
        \includegraphics[scale=0.5]{sim/two_particle/studies/neck_remeshing/step_count}
        \caption{Step Counts and Simulation Time Consumption}
        \label{fig:two_particle/studies/neck_remeshing/step_count}
    \end{subfigure}%
    \begin{subfigure}{0.5\linewidth}
        \includegraphics[scale=0.5]{sim/two_particle/studies/neck_remeshing/shrinkage}
        \caption{Shrinkage}
        \label{fig:two_particle/studies/neck_remeshing/shrinkage}
    \end{subfigure}
    \begin{subfigure}{0.5\linewidth}
        \includegraphics[scale=0.5]{sim/two_particle/studies/neck_remeshing/volume_loss}
        \caption{Volume Loss}
        \label{fig:two_particle/studies/neck_remeshing/volume_loss}
    \end{subfigure}%
    \begin{subfigure}{0.5\linewidth}
        \includegraphics[scale=0.5]{sim/two_particle/studies/neck_remeshing/time_step_width}
        \caption{Time Step Width}
        \label{fig:two_particle/studies/neck_remeshing/time_step_width}
    \end{subfigure}
    \caption{Numerical Behavior of the Simulation in Dependence on the Neck Remeshing Limit}
\end{figure}

\autoref{fig:two_particle/studies/neck_remeshing/step_count} shows the count of calculated time steps alongside with the total time consumption for the simulations.
In general, the step count and time consumption show an exponential dependence on the node count.
Aggressive remeshing (higher limit values) is able to decrease the effort by factors from \numrange{2}{5}.

\autoref{fig:two_particle/studies/neck_remeshing/shrinkage} shows the shrinkage curves obtained from the parameter variation.
In late stages the distinct simulations are in good agreement.
Especially depending on the node count, the agreement in early stages is much weaker.
This can be explained by the disregardance of local geometrical features near the neck when using coarse discretization.
When the gradients are smaller in late stages this influence has less effect.
However, within the node count groups the agreement is good, with low influence of the neck limit parameter.

\autoref{fig:two_particle/studies/time_step/volume_loss} shows the loss of volume in the system during simulation in dependence on the parameters.
As has to be expected, the numerical volume loss shows significant dependence on the node count and remeshing regime.
Coarse discretization and aggressive remeshing lead to higher volume losses.
All stay in the order of \num{1e-3} and thus far below the shrinkage by diffusional flows observed and can therefore be tolerated.

\autoref{fig:two_particle/studies/neck_remeshing/time_step_width} shows the evolution of the time step width used in the simulations for the distinct cases.
The achievable time step size is, as expected, increasing with the reduction of node count.
Aggressive remeshing generally allows higher time steps in later stages and avoids the mentioned decrease in time step width when neck node and the next surface node approach.

\subsection{Summary}

The obtained results show, that if one is interested in the behavior in early stages, a high node count has to be used.
If the regard is only payed on late stages, low node counts lead to equivalent results.
The aggressiveness of neck remeshing is of minor importance regarding the simulation accuracy, but is able to significantly decrease the computational effort.
For the sake of efficiency, as to be expected, node counts as low as possible are preferred, as the effort increases exponentially with the node count.

\section{Study on the Influence of Surface Remeshing}\label{sec:study-surface-remeshing}

The choice of discretization width is always a trade-off between precision of the numerical procedure and it's computational cost.
Usually one aims at the finest discretization required and the coarsest possible.
Often in simulation tasks, this choice has to be made one time before running the simulation.
In the current case, these needs and possibilities highly vary during progress of simulation.
So in the beginning of the sintering process, where curvatures are high a fine discretization is desirable to not miss the details.
In later stages, curvatures have decreased and fine discretization leads to unnecessarily high effort.

This study aims on evaluating the influence of the surface remeshing procedure described in \autoref{sec:remeshing-procedures} on the precision of time integration and providing a decent choice of the parameters to use in the following investigations.

\subsection{Construction of the Study}

The case matrix of the study has two dimensions.
The first is the count of nodes per particle in the initial state, which is inverse proportional to the mean discretization width.
The levels used are \numlist{50;100;200} nodes per particle.
The second is the activation and parameterization of the surface remeshing procedure (\autoref{sec:free-surface-remeshing}).
The levels used are deactivated and the angle thresholds of \qtylist{0.01;0.02;0.05}{\radian}.
As the surface follows a circular shape, high influence of node adding has not to be expected here, so it is disregarded.
The surface distance limit is fixed to the default of \num{3.0}.
Neck remeshing is necessarily activated with a distance limit parameter of \num{0.5}, as was determined as appropriate in \autoref{sec:study-neck-remeshing}.

Targets of investigation are as before the time consumption (duration) of the simulations, the step count needed and the evolution of the step width's during the simulation.
As measure for precision, their agreement in shrinkage and the overall volume loss are regarded.

\subsection{Discussion of the Results}

\begin{figure}
    \begin{subfigure}{0.5\linewidth}
        \includegraphics[scale=0.5]{sim/two_particle/studies/surface_remeshing/step_count}
        \caption{Step Counts and Simulation Time Consumption}
        \label{fig:two_particle/studies/surface_remeshing/step_count}
    \end{subfigure}%
    \begin{subfigure}{0.5\linewidth}
        \includegraphics[scale=0.5]{sim/two_particle/studies/surface_remeshing/shrinkage}
        \caption{Shrinkage}
        \label{fig:two_particle/studies/surface_remeshing/shrinkage}
    \end{subfigure}
    \begin{subfigure}{0.5\linewidth}
        \includegraphics[scale=0.5]{sim/two_particle/studies/surface_remeshing/volume_loss}
        \caption{Volume Loss}
        \label{fig:two_particle/studies/surface_remeshing/volume_loss}
    \end{subfigure}%
    \begin{subfigure}{0.5\linewidth}
        \includegraphics[scale=0.5]{sim/two_particle/studies/surface_remeshing/time_step_width}
        \caption{Time Step Width}
        \label{fig:two_particle/studies/surface_remeshing/time_step_width}
    \end{subfigure}
    \caption{Numerical Behavior of the Simulation in Dependence on the Surface Remeshing Limit}
\end{figure}

\autoref{fig:two_particle/studies/surface_remeshing/step_count} shows the count of calculated time steps alongside with the total time consumption for the simulations.
The general trend is similar to that of neck remeshing discussed in the previous section.
Step count and time consumption increase with node count and decrease with the limit parameter.
A special observation here is the \num{200} nodes/\qty{0.05}{\radian} limit run which shows a drastic drop in the effort compared to the other runs.
The reason for this is, that in this case the limit is high enough to remesh the free surface of the circular shape down to approximately \num{100} nodes in the first remeshing run.
So this case is roughly equivalent to the \num{100} nodes/\qty{0.05}{\radian} limit run.

\autoref{fig:two_particle/studies/surface_remeshing/shrinkage} shows the shrinkage curves obtained from the parameter variation.
As above, the trend is generally equivalent to the neck remeshing.
The low count runs show the same errors in the early stages, whereas the good agreement in late stages is present, too.
Note that the \num{200} nodes/\qty{0.05}{\radian} still shows the accurate results of the other \num{200} nodes runs, despite the agressive remeshing.

\autoref{fig:two_particle/studies/time_step/volume_loss} shows the loss of volume in the system during simulation in dependence on the parameters.
As to be expected, the general trends are similar to the neck remeshing, too.
The special \num{200} nodes/\qty{0.05}{\radian} case shows an extraordinarily low volume loss in late stages, as the effect of neck remeshing losses is canceled out by the counter-signed effect of the initial rigorous surface remeshing step, which was noticed above.

\autoref{fig:two_particle/studies/surface_remeshing/time_step_width} shows the evolution of the time step width used in the simulations for the distinct cases.
As to be expected, the general trends are similar to the neck remeshing, too.
The special \num{200} nodes/\qty{0.05}{\radian} case shows the same bahvior as the other \num{200} nodes cases in the early stages but much higher time steps than those in later stages.

\subsection{Summary}

The results regarding surface remeshing generally show the same trends as with the neck remeshing.
The case of high initial node counts and agressive surface remeshing shows, that it could be preferable to use high initial node counts and use an aggressive remeshing step to create the initial state.
With this procedure fine discretization is only maintained in regions where it is necessary (high curvatures).

\section{Conclusions from Numerical Investigations}\label{sec:numerical-conclusion}

From the previous investigations the overall strategy applied in the following investigations is as follows:
\begin{enumerate}
    \item fine initial discretization to capture the detailed shape features near the neck in the initial states
    \item aggressive remeshing of the free surface to allow for large time steps in late stages
    \item aggressive remeshing of the neck neighborhood to avoid small time steps while running into the singularity
\end{enumerate}
A fourth strategy is derived from these to combine the advantages of fine and coarse discretization where needed:
\begin{enumerate}[resume]
    \item introduce a protected region of about \num{10} nodes around the neck where surface remeshing is suppressed
\end{enumerate}
This last strategy enables taking advantage of coarsening free surfaces by remeshing aggressively but preventing removing too much nodes near the neck to be able to represent the neck geometry precisely.

