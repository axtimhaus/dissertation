\section{Mathematical Description of Diffusion}\label{sec:diffusion}

The classic description of diffusional processes is done by Fick's laws \textcite{Fick1855}, which have the same mathematical form as the heat conduction laws by \textcite{Fourier1888}.
\autoref{eq:fick-first} shows Fick's first law for description of the diffusional flux $\Flux$ for the one-dimensional case.
It proposes a linear relationship between the concentration gradient and the flux.
For the concentration here the vacancy concentration $\VacancyConcentration$ is taken, as we assume only one diffusing species.
The flux of vacancies equals the negative flux of atoms.

\begin{equation}
    \Flux = - \DiffusionCoefficient \frac{\Deriv\VacancyConcentration}{\Deriv\X}
    \label{eq:fick-first}
\end{equation}

The diffusion coefficient heavily varies with temperature.
Its temperature dependence is usually described by an Arrhenius-type equation as in \autoref{eq:diffusion-coefficient-arrhenius} with a pre-exponential factor $\DiffusionCoefficient_0$ and the activation energy $\ActivationEnergy$.

\begin{equation}
    \DiffusionCoefficient = \DiffusionCoefficient_0 \exp \left( -\frac{\ActivationEnergy}{\GasConstant\Temperature} \right)
    \label{eq:diffusion-coefficient-arrhenius}
\end{equation}

The local vacancy concentration as driving force of the diffusion, however, depends on the thermal equilibrium vacancy concentration $\VacancyConcentration^\Standard$ and the local chemical potential $\ChemicalPotential$ as in \autoref{eq:vacancy-concentration}.

\begin{equation}
    \VacancyConcentration = \VacancyConcentration^\Standard \left( 1 - \frac{\ChemicalPotential - \ChemicalPotential^\Standard}{\GasConstant\Temperature} \right)
    \label{eq:vacancy-concentration}
\end{equation}

The main factor influencing the chemical potential in the regarded sintering processes are stresses, especially stresses originating from curved surfaces or interfaces (also known as surface tensions). The local chemical potential by a stress $\Stress$ is given as in \autoref{eq:potential-stress} with the molar volume $\MolarVolume$.

\begin{equation}
    \ChemicalPotential = \ChemicalPotential^\Standard + \Stress \MolarVolume
    \label{eq:potential-stress}
\end{equation}

The stress by a curved surface is given as in \autoref{eq:stress-surface-curvature} with the surface resp.\ interface energy $\InterfaceEnergy$ and the curvature $\Curvature$.

\begin{equation}
    \Stress = \InterfaceEnergy \Curvature
    \label{eq:stress-surface-curvature}
\end{equation}

Other stresses present affect the progress of sintering nevertheless.
The reason for pressure assisted sintering can directly be seen from these equations, since the applied pressure drives the diffusion from the inner of the grain boundaries to the necks.
