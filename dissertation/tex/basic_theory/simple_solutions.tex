\section{Simplified Quantitative Descriptions of Sintering Processes}\label{sec:simple-solutions}

As the mathematical description of diffusion phenomena based on potentials and empirical kinetic diffusion coefficients is quite simple,
the real complexity of modelling sintering processes arises in the geometrical state and evolution of the system.
Early models on particle level therefore posed heavy assumptions on the geometry of the initial particles, as well as of the sinter neck, to be able to obtain a concise and efficient formulation, especially as powerful computer systems were not available.
Such models shall be regarded in the following, more elaborated approaches, however, are regarded in \cref{ch:numerical-methods}.

The most common simplification applied is the assumption of a circular (2D) or spherical (3D) geometry of the particles before sintering.
This has the following main benefits:
\begin{itemize}
    \item simple characterization of particles using a single parameter (radius or diameter)
    \item absence of diffusional flows on large parts of the particle surface due to constant curvature, the only gradients are present near the neck
    \item symmetry of the particle contact with respect to the approaching axis and thus also invariance of particle rotation
\end{itemize}
Such a system's state is sufficiently described by the radii of involved particles, their distance (or positions) and the current size of the necks between the particles.
Often, only a single pair of particles of the same size and material properties is regarded which introduces another degree of symmetry.

These simplifications allow the analytical solution of the governing differential equations and lead to equation of potential type as in \cref{eq:inital-stage-neck-size} for neck radius and \cref{eq:inital-stage-shrinkage} for shrinkage.
Therein, $\Time$ denotes the sinter duration, $\NeckRadius$ the neck radius, $\Radius_{\Particle}$ the particle radius and $B$, $m$ and $n$ are parameters depending on the acting transport mechanisms.
Notable examples of such models are given by \textcite{Kuczynski1949, Johnson1963, Kingery1955, Coblenz1980, German2017}.
\begin{align}
    \left( \frac{\NeckRadius}{\Radius_{\Particle}} \right)^n &= \frac{B \Time}{\left( 2 \Radius_{\Particle} \right)^m}
    \label{eq:inital-stage-neck-size}\\
    \Shrinkage^n &= \frac{B \Time}{\left( 2 \Radius_{\Particle} \right)^m}
    \label{eq:inital-stage-shrinkage}
\end{align}

A common result, also influencing more advanced approaches, from these models alongside with dimensionless parameter studies is the definition of the dimensionless time scale for two-particle-sintering processes as given in \cref{eq:normalized_time}.
Often the surface energy $\InterfaceEnergy_{\Surface}$ and surface diffusion coefficient $\DiffusionCoefficient_{\Surface}$ are referenced, but the equal is possible using the properties of the grain boundary.
\begin{equation}
    \Normalized\Time = \frac{\MolarVolume\VacancyConcentration^\Standard\DiffusionCoefficient_{\Surface}\InterfaceEnergy_{\Surface}}{\GasConstant\Temperature\Radius_{\Particle}^4} \cdot \Time
    \label{eq:normalized_time}
\end{equation}

Similar treatments have been given for the intermediate and final stages of sintering.
In these stages the particles are usually assumed as polyhedra with cylindrical or spherical pores on the boundaries.
Since these stages are mainly characterized by the decrease in pore volume, they are formulated in terms of densification.
Examples for such models are \textcite{Coble1961a} and \textcite{Swinkels1981}.

The predictive value of such simple solutions is rather low because of their heavy simplifications, but they can be used to create empirical fits on experimental data.
