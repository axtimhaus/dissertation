\chapter{Randomized Simulation of Particle Packings}\label{ch:randomized-simulations}

In the previous chapter the numerical and physical behavior of the model was analysed.
Thereby, the transition from a two-particle-model to a multi-particle-model was performed.
The results of \autoref{sec:contact-study} show, that the effects of pore closure can be incoporated by investigating a particle arrangement of at least three particles.
The sintering behavior of all investigated arrangements was the same except for the region where the pore is almost closed.
Also, the effects of multi-material contacts can be regarded.
In the following investigation, the particle arrangement shall be randomized using statistically generated particles as developed in \autoref{ch:statistical-powder-modeling}.

\section{Construction of the Study}

The study will stick to three-particle arrangements to avoid major difficulties in construction of the initial state.
As the size and shape of the particles will now be variable, the initial position can not be given a-priori but must be determined numerically.
The study will compare three cases: three powders with equal particle size distribution but one assumed to have circular particles,
one where only an ovality according to \autoref{eq:f-ovality} is considered
and one where also first order waves according to \autoref{eq:f-first-order-waves} are considered.
The differences and similarities of the three cases will be investigated to rate the need for inclusion of this data into sintering simulations.
The core question is if the first order waves have major influence or if regard on ovality is enough, since ovality is a parameter that can be measured without problems using state of the art automatic particle classifiers.
First order waves, however, require an high amount of manual work.
The author has already published a paper describing a method to obtain these for aggregate particles via laser scanning microscopy images and leat-square fitting in \textcite{Weiner2022a}, but there an another approach of sintering model is used.

\subsection{Statistical Description of the Example Powder}

The values used in this study do not represent a real powder, but are crafted to illustrate the procedure and reveal the influence of the particle shape counter classic circular shapes.
The results will be normalized as was done in the previous studies, so the distributions in the following are designed to support this.

The particle size distribution for all powders is taken as a Weibull distribution (\autoref{eq:weibull-particle-size}) with expectation $\Expectation(\Radius_0)$, which will be used as the reference radius to calculate the characteristic time norm according to \autoref{eq:normalized_time}.
The shape parameter $k$ is taken as \num{3.602} to achieve a non-skewed distribution.
Then, the scale parameter $\lambda$ can directly be obtained from the expectation as given below.
$\Gamma$ is the gamma function.
\begin{equation}
    \Probability(\Radius_0) = \lambda k (\lambda \Radius_0)^{k-1} \exp -(\lambda \Radius_0)^k \qquad \text{with} \qquad \lambda = \frac{\Gamma(1+\frac1k)}{\Expectation(\Radius_0)}
    \label{eq:weibull-particle-size}
\end{equation}

The ovality is modeled using an exponential distribution which is shifted by \num{1} and has the expectation $\Expectation{\Ovality}= \num{1.5}$:
\begin{equation}
    \Probability(\Ovality)  = \frac{1}{\lambda} \exp -\lambda \left( \Ovality - 1 \right) \qquad \text{with} \qquad \lambda = \frac{1}{\Expectation(\Ovality) - 1}
    \label{eq:exponential-ovality}
\end{equation}

The wave height is modeled using a beta distribution:\todo{Explain reasoning of distribution choice.}
\begin{equation}
    \Probability(\WaveHeight) = \frac{1}{\Beta(\alpha, \beta)} \WaveHeight^{\alpha - 1} \left( 1 - \WaveHeight \right) ^ {\beta - 1}
    \label{eq:beta-wave-height}
\end{equation}

\subsection{Creation of the Initial State}

The intial state generation starts by sampling three particles randomly.
For each a radius $\Radius_0$ and one of every shape parameter is sampled from the given distribution.
Also, the rotation of each particle is sampled from a uniform distribution with value range $\RotationAngle \in [0, 2\PI)$.
The initial position of the particles is created in enough distance to each other, so that it is impossible for them to be in contact.

The first particle is kept fixed in space.
Then, the second is moved along the vector towards the center of the first in small steps until contact is created by a defined minimum intersection of the two surfaces.
Then the third is moved along the sum of of the vectors from the thirds center to the others' centers.
If contact with one is created, then the vector towards this particle is counted negative to allow sideward movement.
When the third particle has contact to both, the procedure is completed.

\section{Discussion of the Results}

\todo{Add and discuss results. (Simulation not conducted yet.)}
