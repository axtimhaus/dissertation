\section{Establishing the Equation System}\label{sec:equation-system}

In the following the necessary equations shall be composed.
This includes the definition of the dissipation $\Dissipation$, the dissipation function $\DissipationFunction$, required constraints and additional constraints.
The resulting components of the Lagrangian gradient and the respective Jacobian matrix are omitted here for brevity, but listed in \autoref{app:equation-system}.

\minisec{Dissipation $\Dissipation$}

The dissipation at one node is given by the product of node shifts and the respective Gibbs energy derivatives as in \autoref{eq:dissipation-node}.
The Gibbs energy derivatives are derived in \autoref{sec:surface-evolution} for the different node types.
The dissipation of the whole system is the sum of all node dissipations, since all thermodynamic forces occur at nodes.
Shifting of particles alone does not follow a thermodynamic force, but is determined by the ensemble of the involved nodes.

\begin{equation}
    \Dissipation = -\sum^{\Nodes} \left[ \frac{\partial \GibbsEnergy}{\partial \Shift_{\Normal}} \dot\Shift_{\Normal} + \frac{\partial \GibbsEnergy}{\partial \Shift_{\Tangential}} \dot\Shift_{\Tangential} \right]
    \label{eq:dissipation-node}
\end{equation}

\minisec{Dissipation Function $\DissipationFunction$}

The dissipation function as formulation of the dissipation in terms of fluxes involves the square of fluxes,
the surface distance $\SurfaceDistance_{\Upper}$,
the diffusion coefficient along the interface $\DiffusionCoefficient_{\Upper}$,
the interface thickness $\InterfaceThickness_{\Upper}$,
the molar volume $\MolarVolume$ of the particle substance,
the thermal vacancy concentration $\VacancyConcentration^{\Standard}$ at the respective temperature,
the universal gas constant $\GasConstant$
and the absolute temperature $\Temperature$.
Regarding only the fluxes to the upper node is sufficient, since the fluxes to the lower are regarded from the lower node on.

\begin{equation}
    \DissipationFunction = \sum^{\Nodes} \frac{\GasConstant\Temperature}{\MolarVolume{\VacancyConcentration}^\Standard} \frac{\SurfaceDistance_{\Upper} {\Flux}_{\Upper}^2}{{\InterfaceThickness_{\Upper}\DiffusionCoefficient}_{\Upper}}
    \label{eq:dissipation-function-node}
\end{equation}

The Langrangian multiplicator for the equality constraint of both dissipation formulations (\autoref{eq:constraint-dissipation-equality}) will be denoted as $\LagrangeParameter_{\Dissipation}$ in the following.

\begin{equation}
    \Dissipation = \DissipationFunction
    \label{eq:constraint-dissipation-equality}
\end{equation}

\minisec{Required Constraints $\Vect\RequiredConstraint$}

The required constraints needed to link internal state velocities and fluxes are given by the constancy of volume respectively mass at each node.
The volume change at the node given by its displacement must be equal to the divergence of fluxes from and to this node.
The partial derivatives of volume will be derived in \autoref{sec:surface-evolution}.

\begin{equation}
    \frac{\partial\Volume}{\partial{\Shift}_{\Normal}} {\dot\Shift}_{\Normal} + \frac{\partial\Volume}{\partial{\Shift}_{\Tangential}} {\dot\Shift}_{\Tangential} = \dot\Volume = \left( {\Flux}_{\Upper} + {\Flux}_{\Lower} \right)
    \label{eq:required-constraint-flux-shift}
\end{equation}

The Langrangian multiplicator for this constraint will be denoted as $\LagrangeParameter_\Volume$ in the following.

\minisec{Additional Constraints $\Vect\AdditionalConstraint$}

To fix a group of particles' position in space, one of them has to be constrained at a fixed position, so that the other ones are able to move relatively to this one.
This is done by a constraints setting it's displacement to zero as in \autoref{eq:fixed-particle-constraints}.

\begin{subequations}
    \begin{align}
        \Step\X_{\Particle} = 0 \\
        \Step\Y_{\Particle} = 0
    \end{align}
    \label{eq:fixed-particle-constraints}
\end{subequations}

To ensure the validity of the grain boundaries (lack of voids and overlaps), contact constraints are to be applied to each pair of contact nodes, which must retain the identical position in global space as discussed in \autoref{sec:particle-representation}.
These constraints are developed in \autoref{sec:contact-conditions}.

Further constraints of the like are possible, for example to enforce a certain particle movement or apply an external load on a particle to simulate assisted compaction.
