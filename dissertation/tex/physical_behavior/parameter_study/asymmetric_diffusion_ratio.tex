\subsection{Asymmetric Diffusion Ratio \texorpdfstring{$\DiffusionCoefficient_{2} / \DiffusionCoefficient_{1}$}{D2/D1}}\label{subsec:parameter-study-asymmetric-surface-diffusion-ratio}

\begin{figure}
    \begin{subfigure}{\linewidth}
        \centering
        \includegraphics{sim/two_particle/studies/diffusion_asymmetric/10.00000/evolution}
        \caption{$\DiffusionCoefficient_{2} / \DiffusionCoefficient_{1} = \num{10}$}
        \label{fig:two_particle/studies/diffusion_asymmetric/10.00000/evolution}
    \end{subfigure}
    \begin{subfigure}{\linewidth}
        \centering
        \includegraphics{sim/two_particle/studies/diffusion_asymmetric/100.00000/evolution}
        \caption{$\DiffusionCoefficient_{2} / \DiffusionCoefficient_{1} = \num{100}$}
        \label{fig:two_particle/studies/diffusion_asymmetric/100.00000/evolution}
    \end{subfigure}
    \caption{Geometry Evolutions for Selected Asymmetric Surface Diffusion Ratios}
    \label{fig:two_particle/studies/diffusion_asymmetric/evolution}
\end{figure}

\begin{figure}
    \begin{subfigure}{\linewidth}
        \centering
        \includegraphics{sim/two_particle/studies/diffusion_asymmetric/neck_size}
        \caption{Evolution Curves}
        \label{fig:two_particle/studies/diffusion_asymmetric/neck_size}
    \end{subfigure}
    \begin{subfigure}{\linewidth}
        \centering
        \includegraphics{sim/two_particle/studies/diffusion_asymmetric/neck_size_map}
        \caption{Time Isolines}
        \label{fig:two_particle/studies/diffusion_asymmetric/neck_size_map}
    \end{subfigure}
    \caption{Neck Size in Dependence on Asymmetric Surface Diffusion Ratio}
\end{figure}

\begin{figure}
    \begin{subfigure}{\linewidth}
        \centering
        \includegraphics{sim/two_particle/studies/diffusion_asymmetric/shrinkage}
        \caption{Evolution Curves}
        \label{fig:two_particle/studies/diffusion_asymmetric/shrinkage}
    \end{subfigure}
    \begin{subfigure}{\linewidth}
        \centering
        \includegraphics{sim/two_particle/studies/diffusion_asymmetric/shrinkage_map}
        \caption{Time Isolines}
        \label{fig:two_particle/studies/diffusion_asymmetric/shrinkage_map}
    \end{subfigure}
    \caption{Shrinkage in Dependence on Asymmetric Surface Diffusion Ratio}
\end{figure}

This study investigates the effect of sintering particles of different substance by means of a difference in their diffusion rate.
The diffusion rates on surface and grain boundary of one particle are taken equal, but vary compared to the other particle.
The initial particle geometry and remaining material parameters are taken equal on both particles with $\InterfaceEnergy_{\GrainBoundary} = \num{0.5}\InterfaceEnergy_{\Surface}$.

\Cref{fig:two_particle/studies/diffusion_asymmetric/evolution} shows two selected evolutions of the contact geometry.
Note especially the increasing curvature of the grain boundary with increasing diffusion ratio, as the contribution by diffusion from the second particle is higher.
This allows the second particle to creep around the first one.
The surface of the second particle is flatter as diffusion is able to equalize it faster.
The shape of the grain boundary clearly shows a limitation of the model, since currently only one grain boundary node is supported (see \cref{sec:contact-conditions}).
The coarsely discretized grain boundary poorly represents the real conditions, where the original surface shape of the less active particle would be approximately maintained while the active one creeps around.

\Cref{fig:two_particle/studies/diffusion_asymmetric/neck_size} shows the dependence of the neck size and \Cref{fig:two_particle/studies/diffusion_asymmetric/shrinkage} the dependence of the shrinkage on the asymmetric surface diffusion ratio.
Both are increasing with the diffusion ratio, as the contribution of the active particle in terms of surface as well grain boundary fluxes is increased.
