\subsection{Ovality \texorpdfstring{$\Ovality$}{o}}\label{subsec:parameter-study-ovality}

This study investigates the influence of non-ideal particle geometry by ovality on the sintering behavior.
As oval particles have two additional degrees of freedom regarding the intial contact by rotation, in the following three distinct cases will be selected.
First the contact of both particles on their tip side (highest curvature side), second of both on their flank side (lowest curvatur side).
The third will be the contact of flank counter tip.

\subsubsection{Tip-Tip-Contact}

\begin{figure}[p]
    \begin{subfigure}{0.5\linewidth}
        \includegraphics[scale=0.5]{sim/two_particle/studies/ovality_tip_tip/2.00000/evolution}
        \caption{$\Ovality = \num{2}$}
        \label{fig:two_particle/studies/ovality_tip_tip/2.00000/evolution}
    \end{subfigure}%
    \begin{subfigure}{0.5\linewidth}
        \includegraphics[scale=0.5]{sim/two_particle/studies/ovality_tip_tip/3.00000/evolution}
        \caption{$\Ovality = \num{3}$}
        \label{fig:two_particle/studies/ovality_tip_tip/3.00000/evolution}
    \end{subfigure}
    \caption{Geometry Evolutions for Selected Ovalities in Tip-Tip Contacts}
    \label{fig:two_particle/studies/ovality_tip_tip/evolution}
\end{figure}

\begin{figure}[p]
    \begin{subfigure}{0.5\linewidth}
        \includegraphics[scale=0.5]{sim/two_particle/studies/ovality_tip_tip/neck_size}
        \caption{Evolution Curves}
        \label{fig:two_particle/studies/ovality_tip_tip/neck_size/a}
    \end{subfigure}%
    \begin{subfigure}{0.5\linewidth}
        \includegraphics[scale=0.5]{sim/two_particle/studies/ovality_tip_tip/neck_size_map}
        \caption{Time Isolines}
        \label{fig:two_particle/studies/ovality_tip_tip/neck_size/b}
    \end{subfigure}
    \caption{Neck Size in Dependence on Ovality in Tip-Tip Contacts}
    \label{fig:two_particle/studies/ovality_tip_tip/neck_size}
\end{figure}

\begin{figure}[p]
    \begin{subfigure}{0.5\linewidth}
        \includegraphics[scale=0.5]{sim/two_particle/studies/ovality_tip_tip/shrinkage}
        \caption{Evolution Curves}
        \label{fig:two_particle/studies/ovality_tip_tip/shrinkage/a}
    \end{subfigure}%
    \begin{subfigure}{0.5\linewidth}
        \includegraphics[scale=0.5]{sim/two_particle/studies/ovality_tip_tip/shrinkage_map}
        \caption{Time Isolines}
        \label{fig:two_particle/studies/ovality_tip_tip/shrinkage/b}
    \end{subfigure}
    \caption{Shrinkage in Dependence on Ovality in Tip-Tip Contacts}
    \label{fig:two_particle/studies/ovality_tip_tip/shrinkage}
\end{figure}

\Cref{fig:two_particle/studies/ovality_tip_tip/evolution} shows two selected evolutions of the contact geometry.
The plots clearly show the limitation of neck growth by the small diameter of the ellipse.
The fill volume needed to grow the neck is increased by ovality due to the wide surface profile around the neck.

\Cref{fig:two_particle/studies/ovality_tip_tip/neck_size} shows the dependence of the neck size on the particle ovality.
The neck size heavily decreases with the ovality as the fill volume is increased and the neck growth is limited by the small diameter of the ellipse.
The oval particles behaves here similarly to a particle of smaller size.

\Cref{fig:two_particle/studies/ovality_tip_tip/shrinkage} shows the dependence of the shrinkage on the particle ovality.
The shrinkage slightly increases with the ovality.
There are two counter-acting effects here.
First, the diffusion paths are shorter due to the small neck.
Second, the distance between the particle centers is increased due to the oval shape, which increases the denominator of shrinkage.
The latter influence dominates apparently.

\subsubsection{Flank-Flank-Contact}

\begin{figure}[p]
    \begin{subfigure}{0.5\linewidth}
        \includegraphics[scale=0.5]{sim/two_particle/studies/ovality_flank_flank/2.00000/evolution}
        \caption{$\Ovality = \num{2}$}
        \label{fig:two_particle/studies/ovality_flank_flank/2.00000/evolution}
    \end{subfigure}%
    \begin{subfigure}{0.5\linewidth}
        \includegraphics[scale=0.5]{sim/two_particle/studies/ovality_flank_flank/3.00000/evolution}
        \caption{$\Ovality = \num{3}$}
        \label{fig:two_particle/studies/ovality_flank_flank/3.00000/evolution}
    \end{subfigure}
    \caption{Geometry Evolutions for Selected Ovalities in Flank-Flank Contacts (Time Increasing from Violet to Yellow)}
    \label{fig:two_particle/studies/ovality_flank_flank/evolution}
\end{figure}

\begin{figure}[p]
    \begin{subfigure}{0.5\linewidth}
        \includegraphics[scale=0.5]{sim/two_particle/studies/ovality_flank_flank/neck_size}
        \caption{Evolution Curves}
        \label{fig:two_particle/studies/ovality_flank_flank/neck_size/a}
    \end{subfigure}%
    \begin{subfigure}{0.5\linewidth}
        \includegraphics[scale=0.5]{sim/two_particle/studies/ovality_flank_flank/neck_size_map}
        \caption{Time Isolines}
        \label{fig:two_particle/studies/ovality_flank_flank/neck_size/b}
    \end{subfigure}
    \caption{Neck Size in Dependence on Ovality in Flank-Flank Contacts}
    \label{fig:two_particle/studies/ovality_flank_flank/neck_size}
\end{figure}

\begin{figure}[p]
    \begin{subfigure}{0.5\linewidth}
        \includegraphics[scale=0.5]{sim/two_particle/studies/ovality_flank_flank/shrinkage}
        \caption{Evolution Curves}
        \label{fig:two_particle/studies/ovality_flank_flank/shrinkage/a}
    \end{subfigure}%
    \begin{subfigure}{0.5\linewidth}
        \includegraphics[scale=0.5]{sim/two_particle/studies/ovality_flank_flank/shrinkage_map}
        \caption{Time Isolines}
        \label{fig:two_particle/studies/ovality_flank_flank/shrinkage/b}
    \end{subfigure}
    \caption{Shrinkage in Dependence on Ovality in Flank-Flank Contacts}
    \label{fig:two_particle/studies/ovality_flank_flank/shrinkage}
\end{figure}

\Cref{fig:two_particle/studies/ovality_flank_flank/evolution} shows two selected evolutions of the contact geometry.
The plots clearly show the limitation of neck growth by the large diameter of the ellipse.
The fill volume is decreased by ovality due to the narrow gap between the particle surfaces.

\Cref{fig:two_particle/studies/ovality_flank_flank/neck_size} shows the dependence of the neck size on the particle ovality.
Neck size is heavily increasing with ovality, as the volume to fill is small and it is here limited by the large diameter of the ellipse.
The particles behave like larger ones.

\Cref{fig:two_particle/studies/ovality_flank_flank/shrinkage} shows the dependence of the shrinkage on the particle ovality.
The shrinkage shows a slight decrease with the ovality.
The reasoning is contrary to the tip-tip case, as here longer diffusion paths and smaller particle distance are in effect.
The length of the diffusion path appears to dominate.

\subsubsection{Tip-Flank-Contact}

\begin{figure}[p]
    \begin{subfigure}{0.5\linewidth}
        \includegraphics[scale=0.5]{sim/two_particle/studies/ovality_tip_flank/2.00000/evolution}
        \caption{$\Ovality = \num{2}$}
        \label{fig:two_particle/studies/ovality_tip_flank/2.00000/evolution}
    \end{subfigure}%
    \begin{subfigure}{0.5\linewidth}
        \includegraphics[scale=0.5]{sim/two_particle/studies/ovality_tip_flank/3.00000/evolution}
        \caption{$\Ovality = \num{3}$}
        \label{fig:two_particle/studies/ovality_tip_flank/3.00000/evolution}
    \end{subfigure}
    \caption{Geometry Evolutions for Selected Ovalities in Tip-Flank Contacts (Time Increasing from Violet to Yellow)}
    \label{fig:two_particle/studies/ovality_tip_flank/evolution}
\end{figure}

\begin{figure}[p]
    \begin{subfigure}{0.5\linewidth}
        \includegraphics[scale=0.5]{sim/two_particle/studies/ovality_tip_flank/neck_size}
        \caption{Evolution Curves}
        \label{fig:two_particle/studies/ovality_tip_flank/neck_size/a}
    \end{subfigure}%
    \begin{subfigure}{0.5\linewidth}
        \includegraphics[scale=0.5]{sim/two_particle/studies/ovality_tip_flank/neck_size_map}
        \caption{Time Isolines}
        \label{fig:two_particle/studies/ovality_tip_flank/neck_size/b}
    \end{subfigure}
    \caption{Neck Size in Dependence on Ovality in Tip-Flank Contacts}
    \label{fig:two_particle/studies/ovality_tip_flank/neck_size}
\end{figure}

\begin{figure}[p]
    \begin{subfigure}{0.5\linewidth}
        \includegraphics[scale=0.5]{sim/two_particle/studies/ovality_tip_flank/shrinkage}
        \caption{Evolution Curves}
        \label{fig:two_particle/studies/ovality_tip_flank/shrinkage/a}
    \end{subfigure}%
    \begin{subfigure}{0.5\linewidth}
        \includegraphics[scale=0.5]{sim/two_particle/studies/ovality_tip_flank/shrinkage_map}
        \caption{Time Isolines}
        \label{fig:two_particle/studies/ovality_tip_flank/shrinkage/b}
    \end{subfigure}
    \caption{Shrinkage in Dependence on Ovality in Tip-Flank Contacts}
    \label{fig:two_particle/studies/ovality_tip_flank/shrinkage}
\end{figure}

\Cref{fig:two_particle/studies/ovality_tip_flank/evolution} shows two selected evolutions of the contact geometry.
The neck size is limited by the the small diameter of the ellipse.
The grain boundary is heavily curved due to surface fluxes of the particle in flank position.
This would lead to consumption of the tip-positioned particle by the flank-positioned one if cross-boundary fluxes were allowed by the model.
So, although the particles are of equal size, the contact resembles that of unequally sized particles.

\Cref{fig:two_particle/studies/ovality_tip_flank/neck_size} shows the dependence of the neck size on the particle ovality.
As in the tip-tip case, neck size decreases with ovality, but much less pronounced.
Also larger neck widths compared to tip-tip are possible, the neck may even grow larger than the small ellipse diameter.

\Cref{fig:two_particle/studies/ovality_tip_flank/shrinkage} shows the dependence of the shrinkage on the particle ovality.
The shrinkage increases slightly with the ovality similarly to the tip-tip case.
Although the shorter diffusion paths are acting here, too, the initial distance stays constant over the whole parameter range since the increase by the tip-positioned particle is equalized by the decrease by the flank-positioned one.

\subsubsection{Summary}

The influence of ovality on shrinkage is to be rated as rather low.
Depending on the contact configuration their is only a slight increase or decrease with the ovality.
Contrary, the effect on neck size is large.
The particles behave there like smaller or larger particles depending on the contact configuration.
The notion of an apparent radius can be introduced, which is the radius of an equivalent circular particle in regard on neck size evolution.
The neck size is always limited by the smaller of the apparent particle radii.
