\subsubsection{Surface-Boundary Energy Ratio $\InterfaceEnergy_{\GrainBoundary} / \InterfaceEnergy_{\Surface}$}\label{subsubsec:parameter-study-surface-boundary-energy}

\begin{figure}[p]
    \begin{subfigure}{0.5\linewidth}
        \includegraphics[scale=0.5]{sim/two_particle/studies/surface_boundary_energy/shrinkage}
        \caption{Evolution Curves}
        \label{fig:two_particle/studies/surface_boundary_energy/shrinkage}
    \end{subfigure}%
    \begin{subfigure}{0.5\linewidth}
        \includegraphics[scale=0.5]{sim/two_particle/studies/surface_boundary_energy/shrinkage_map}
        \caption{Time Isolines}
        \label{fig:two_particle/studies/surface_boundary_energy/shrinkage_map}
    \end{subfigure}
    \caption{Shrinkage in Dependence on Surface-Boundary Energy Ratio}
\end{figure}

\begin{figure}[p]
    \begin{subfigure}{0.5\linewidth}
        \includegraphics[scale=0.5]{sim/two_particle/studies/surface_boundary_energy/neck_size}
        \caption{Evolution Curves}
        \label{fig:two_particle/studies/surface_boundary_energy/neck_size}
    \end{subfigure}%
    \begin{subfigure}{0.5\linewidth}
        \includegraphics[scale=0.5]{sim/two_particle/studies/surface_boundary_energy/neck_size_map}
        \caption{Time Isolines}
        \label{fig:two_particle/studies/surface_boundary_energy/neck_size_map}
    \end{subfigure}
    \caption{Neck Size in Dependence on Surface-Boundary Energy Ratio}
\end{figure}

\begin{figure}[p]
    \begin{subfigure}{0.5\linewidth}
        \includegraphics[scale=0.5]{sim/two_particle/studies/surface_boundary_energy/0.10000/evolution}
        \caption{$\InterfaceEnergy_{\GrainBoundary} / \InterfaceEnergy_{\Surface} = \num{0.1}$}
        \label{fig:two_particle/studies/surface_boundary_energy/0.10000/evolution}
    \end{subfigure}%
    \begin{subfigure}{0.5\linewidth}
        \includegraphics[scale=0.5]{sim/two_particle/studies/surface_boundary_energy/0.90000/evolution}
        \caption{$\InterfaceEnergy_{\GrainBoundary} / \InterfaceEnergy_{\Surface} = \num{0.9}$}
        \label{fig:two_particle/studies/surface_boundary_energy/0.90000/evolution}
    \end{subfigure}
    \caption{Geometry Evolutions for Selected Surface-Boundary Energy Ratios}
    \label{fig:two_particle/studies/surface_boundary_energy/evolution}
\end{figure}

This study investigates the effect of the ratio between grain boundary and surface energy.
The initial particle geometry and the material parameters are taken equal on both particles.

\autoref{fig:two_particle/studies/surface_boundary_energy/shrinkage} shows the dependence of the shrinkage on the interface energy ratio.
With increasing energy ratio the shrinkage is decreased and tends to flatten at earlier times.
The similar holds for the neck size as is shown in \autoref{fig:two_particle/studies/surface_boundary_energy/neck_size}.
Both is due to the decreased driving force as the interface energy ratio increases.

\autoref{fig:two_particle/studies/surface_boundary_energy/evolution} shows two selected evolutions of the contact geometry.
Note the large dihedral angle at low interface energy ratios and the small one at high ratios, since in the latter case the local staionary state around the neck is reached earlier.

\todo{Compare against literature results.}
