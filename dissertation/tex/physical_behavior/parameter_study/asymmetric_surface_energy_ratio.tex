\subsubsection{Asymmetric Surface Energy Ratio $\InterfaceEnergy_{\Surface2} / \InterfaceEnergy_{\Surface1}$}\label{subsubsec:parameter-study-asymmetric-surface-energy-ratio}

\begin{figure}[p]
    \begin{subfigure}{0.5\linewidth}
        \includegraphics[scale=0.5]{sim/two_particle/studies/surface_energy_asymmetric/shrinkage}
        \caption{Evolution Curves}
        \label{fig:two_particle/studies/surface_energy_asymmetric/shrinkage}
    \end{subfigure}%
    \begin{subfigure}{0.5\linewidth}
        \includegraphics[scale=0.5]{sim/two_particle/studies/surface_energy_asymmetric/shrinkage_map}
        \caption{Time Isolines}
        \label{fig:two_particle/studies/surface_energy_asymmetric/shrinkage_map}
    \end{subfigure}
    \caption{Shrinkage in Dependence on Asymmetric Surface Energy Ratio}
\end{figure}

\begin{figure}[p]
    \begin{subfigure}{0.5\linewidth}
        \includegraphics[scale=0.5]{sim/two_particle/studies/surface_energy_asymmetric/neck_size}
        \caption{Evolution Curves}
        \label{fig:two_particle/studies/surface_energy_asymmetric/neck_size}
    \end{subfigure}%
    \begin{subfigure}{0.5\linewidth}
        \includegraphics[scale=0.5]{sim/two_particle/studies/surface_energy_asymmetric/neck_size_map}
        \caption{Time Isolines}
        \label{fig:two_particle/studies/surface_energy_asymmetric/neck_size_map}
    \end{subfigure}
    \caption{Neck Size in Dependence on Asymmetric Surface Energy Ratio}
\end{figure}

\begin{figure}[p]
    \begin{subfigure}{0.5\linewidth}
        \includegraphics[scale=0.5]{sim/two_particle/studies/surface_energy_asymmetric/2.00000/evolution}
        \caption{$\InterfaceEnergy_{\Surface2} / \InterfaceEnergy_{\Surface1} = \num{2}$}
        \label{fig:two_particle/studies/surface_energy_asymmetric/2.00000/evolution}
    \end{subfigure}%
    \begin{subfigure}{0.5\linewidth}
        \includegraphics[scale=0.5]{sim/two_particle/studies/surface_energy_asymmetric/3.00000/evolution}
        \caption{$\InterfaceEnergy_{\Surface2} / \InterfaceEnergy_{\Surface1} = \num{3}$}
        \label{fig:two_particle/studies/surface_energy_asymmetric/3.00000/evolution}
    \end{subfigure}
    \caption{Geometry Evolutions for Selected Asymmetric Surface Energy Ratios}
    \label{fig:two_particle/studies/surface_energy_asymmetric/evolution}
\end{figure}

This study investigates the effect of sintering particles of different substance by means of a difference in their surface energy.
The initial particle geometry and remaining material parameters are taken equal on both particles with an grain boundary energy of $\InterfaceEnergy_{\GrainBoundary} = \num{0.5}\InterfaceEnergy_{\Surface}$.
The study is limited to ratios up to $\InterfaceEnergy_{\Surface2}/\InterfaceEnergy_{\Surface1} = \num{3}$, because above this point the model showed numerical instability as will be discussed below with the geometric evolutions.

\autoref{fig:two_particle/studies/surface_energy_asymmetric/neck_size} shows the dependence of the neck size on the asymmetric surface energy ratio.

\autoref{fig:two_particle/studies/surface_energy_asymmetric/shrinkage} shows the dependence of the shrinkage on the asymmetric surface energy ratio.

\autoref{fig:two_particle/studies/surface_energy_asymmetric/evolution} shows two selected evolutions of the contact geometry.
Note especially the undercut occurring in \autoref{fig:two_particle/studies/surface_energy_asymmetric/evolution} near the neck on the surface of particle 1.
Due to the high surface energy of the second particle it is thermodynamical beneficial to grow the neck while creating additional surface of the first particle.
As the surface diffusion is to slow to deliver enough volume to the neck, an undercut evolves.
Due to the mathematical construction of the model volume can be removed from a surface node in such an undercut despite there is too few left between the surface and the grain boundary, so after a critical time step, surface line and grain boundary cross.
This happens on energy ratios $\InterfaceEnergy_{\Surface2}/\InterfaceEnergy_{\Surface1} > \num{3}$ as mentioned above.
This is of course physically impossible, but at the time of writing this thesis, the issue was not solved.
