\section{Study on the Influence of Key Physical Parameters}\label{sec:parameter-study}

The simulation of an asymmetrical two-particle-contact involves a variety of physical geometry and material parameters which highly influence the sintering behavior and macroscopic response of the system to be observed.
A widely used approach to characterize a system is by it's dimensionless parameters.
The following study investigates the influence of the dimensionless parameters present in an asymmetrical two-particle-system.

\subsection{Construction of the Study}

The shape of a particle is here modeled as an ellipse an parameterized with the equivalent circular radius $\Radius_{\Particle}$ and the ovality $\Ovality \in [ 1, \infty ) $ (as the standard excentricity is an hardly illustrative value for the current purpose).
The local polar radius coordinate of a node is then given by \autoref{eq:parameter_study_ellipse_function}, with the main radii defined by $a = \Radius_{\Particle} \sqrt{\Ovality}$ and $b = \Radius_{\Particle} / \sqrt{\Ovality}$ with $a/b = o$.

\begin{equation}
    \Radius(\Angle) = \frac{a b}{\sqrt{\left( a \sin \Angle \right)^2 + \left( b \cos \Angle \right)^2  }}
    \label{eq:parameter_study_ellipse_function}
\end{equation}

\begin{table}
    \caption{Investigated Dimensionless Parameters and their Range}
    \label{tab:parameter_study_conditions}
    \begin{tblr}{colspec={lrrlX}, row{1}={l}}
        \toprule
        Parameter & Min & Max & Scale & remarks \\
        \midrule
        $\Radius_1 / \Radius_2$ & \num{1} & \num{10} & lin & equal material\\
        $\DiffusionCoefficient_{\GrainBoundary} / \DiffusionCoefficient_{\Surface}$  & \num{0.01} & \num{1} & log & equal material\\
        $\InterfaceEnergy_{\GrainBoundary} / \InterfaceEnergy_{\Surface}$ & \num{0.1} & \num{0.9} & lin & equal material \\
        $\DiffusionCoefficient_{\Surface2} / \DiffusionCoefficient_{\Surface1} $ & \num{1} & \num{100} & log & $\InterfaceEnergy_{\GrainBoundary} / \InterfaceEnergy_{\Surface1} = \num{0.5}$, $\DiffusionCoefficient_{\GrainBoundary} / \DiffusionCoefficient_{\Surface1} = \num{1}$ \\
        $\InterfaceEnergy_{\Surface2} / \InterfaceEnergy_{\Surface1}$ & \num{1} & \num{10} & lin & $\InterfaceEnergy_{\GrainBoundary} / \InterfaceEnergy_{\Surface1} = \num{0.5}$, $\DiffusionCoefficient_{\GrainBoundary} / \DiffusionCoefficient_{\Surface1} = \num{1}$ \\
        $o$ & \num{1} & \num{6} & lin & tip/tip contact, equal material \\
        $o$ & \num{1} & \num{6} & lin & tip/flank contact, equal material \\
        $o$ & \num{1} & \num{6} & lin & flank/flank contact, equal material \\
        \bottomrule
    \end{tblr}
\end{table}

\autoref{tab:parameter_study_conditions} lists the varied dimensionless parameters and their respective value ranges, where the indices 1 and 2 denote the first resp. the second particle.
The study includes:
\begin{description}
    \item[symmetrical contacts] with varied ratios of interface energies and diffusion coefficients between surface and grain boundary
    \item[asymmetrical geometry contacts] with circular particles of different sizes and oval particles in different contact modes
    \item[asymmetrical material contacts] with varied ratios of interface energies and diffusion coefficients on both sides
\end{description}

The initial contact of the particles is always taken in that way, that the distance of the particles equals \qty{1}{\percent} of the reference particle's radius $\Radius_1$.
Then, the intersection of the particle's is calculated and the grain boundary is placed as a straight line between the surface line intersection points.

\subsection{Discussion of the Results}

The results will be displayed as isoline-plots for shrinkage and neck size with the normalized time on the horizontal axis and the regarded dimensionless parameter on the vertical axis.

\begin{frame}{Geometry Evolution -- Particle-Size-Ratio $\Radius_2 / \Radius_1$}
    \centering
    \only<1>{%
        \includegraphics[width=\linewidth]{sim/two_particle/studies/particle_size_ratio/1.00000/evolution}\\
        $\Radius_2 / \Radius_1 = \num{1}$%
    }%
    \only<2>{%
        \includegraphics[width=\linewidth]{sim/two_particle/studies/particle_size_ratio/2.00000/evolution}\\
        $\Radius_2 / \Radius_1 = \num{2}$%
    }%
    \only<3>{%
        \includegraphics[width=\linewidth]{sim/two_particle/studies/particle_size_ratio/3.00000/evolution}\\
        $\Radius_2 / \Radius_1 = \num{3}$%
    }%
    \only<4>{%
        \includegraphics[width=\linewidth]{sim/two_particle/studies/particle_size_ratio/10.00000/evolution}\\
        $\Radius_2 / \Radius_1 = \num{10}$%
    }%
\end{frame}

\subsection{Surface-Boundary Diffusion Ratio \texorpdfstring{$\DiffusionCoefficient_{\GrainBoundary} / \DiffusionCoefficient_{\Surface}$}{DGB/DS}}\label{subsec:parameter-study-surface-boundary-diffusion}

\begin{figure}[p]
    \begin{subfigure}{0.5\linewidth}
        \includegraphics[scale=0.5]{sim/two_particle/studies/surface_boundary_diffusion/0.01000/evolution}
        \caption{$\DiffusionCoefficient_{\GrainBoundary} / \DiffusionCoefficient_{\Surface} = \num{0.01}$}
        \label{fig:two_particle/studies/surface_boundary_diffusion/0.01000/evolution}
    \end{subfigure}%
    \begin{subfigure}{0.5\linewidth}
        \includegraphics[scale=0.5]{sim/two_particle/studies/surface_boundary_diffusion/1.00000/evolution}
        \caption{$\DiffusionCoefficient_{\GrainBoundary} / \DiffusionCoefficient_{\Surface} = \num{1.0}$}
        \label{fig:two_particle/studies/surface_boundary_diffusion/1.00000/evolution}
    \end{subfigure}
    \caption{Geometry Evolutions for Selected Surface-Boundary Diffusion Ratios}
    \label{fig:two_particle/studies/surface_boundary_diffusion/evolution}
\end{figure}

\begin{figure}[p]
    \begin{subfigure}{0.5\linewidth}
        \includegraphics[scale=0.5]{sim/two_particle/studies/surface_boundary_diffusion/shrinkage}
        \caption{Evolution Curves}
        \label{fig:two_particle/studies/surface_boundary_diffusion/shrinkage}
    \end{subfigure}%
    \begin{subfigure}{0.5\linewidth}
        \includegraphics[scale=0.5]{sim/two_particle/studies/surface_boundary_diffusion/shrinkage_map}
        \caption{Time Isolines}
        \label{fig:two_particle/studies/surface_boundary_diffusion/shrinkage_map}
    \end{subfigure}
    \caption{Shrinkage in Dependence on Surface-Boundary Diffusion Ratio}
\end{figure}

\begin{figure}[p]
    \begin{subfigure}{0.5\linewidth}
        \includegraphics[scale=0.5]{sim/two_particle/studies/surface_boundary_diffusion/neck_size}
        \caption{Evolution Curves}
        \label{fig:two_particle/studies/surface_boundary_diffusion/neck_size}
    \end{subfigure}%
    \begin{subfigure}{0.5\linewidth}
        \includegraphics[scale=0.5]{sim/two_particle/studies/surface_boundary_diffusion/neck_size_map}
        \caption{Time Isolines}
        \label{fig:two_particle/studies/surface_boundary_diffusion/neck_size_map}
    \end{subfigure}
    \caption{Neck Size in Dependence on Surface-Boundary Diffusion Ratio}
\end{figure}

This study investigates the effect of the ratio between grain boundary and surface diffusion rate.
This parameter is the key factor on the relation between neck size evolution and shrinkage, as grain boundary diffusion is the only mechanism present in this model able to remove volume from between the particles and thus effect shrinkage.
The initial particle geometry and the material parameters are taken equal on both particles.

\Cref{fig:two_particle/studies/surface_boundary_diffusion/evolution} shows two selected evolutions of the contact geometry.
One may directly observe the lack of noticeable particle movement at low diffusion ratios and the difference in neck size obtained depending on the contribution of boundary diffusion.

\Cref{fig:two_particle/studies/surface_boundary_diffusion/shrinkage} shows the dependence of the shrinkage on the diffusion ratio.
The shrinkage largely varies with the diffusion ratio as the grain boundary diffusion is the only mechanism able to effect shrinkage.
The higher the boundary diffusion coefficient, the higher the shrinkage.
Although, the benefit from accelerated grain boundary diffusion decreases at higher ratios (when grain boundary diffusion is faster than surface diffusion).
The latter case is of less practical relevance, since usually surface diffusion coefficients are much higher in high temperature processes due to their smaller activation energy \cite{Fisher1951}.
The evolution curves do principally show a parallel displacement with the diffusion ratio, but with a noticeable approach in later stages.
Both are likely due to the increased neck size (larger diffusion distance) when boundary diffusion is present.

\Cref{fig:two_particle/studies/surface_boundary_diffusion/neck_size} shows the dependence of the neck size on the diffusion ratio.
The neck growth is heavily affected by the diffusion ratio, too.
When shrinkage occurs due to boundary diffusion, the volume needed to fill the neck is lowered due to the particles' approaching, which increases the speed of neck growth.
In late stages with high grain boundary diffusion the effect flattens again, as the stationary state of the neck is reached.

\subsection{Surface-Boundary Energy Ratio \texorpdfstring{$\InterfaceEnergy_{\GrainBoundary} / \InterfaceEnergy_{\Surface}$}{γGB/γS}}\label{subsec:parameter-study-surface-boundary-energy}

\begin{figure}
    \begin{subfigure}{\linewidth}
        \centering
        \includegraphics[width=\linewidth]{sim/two_particle/studies/surface_boundary_energy/0.10000/evolution}
        \caption{$\InterfaceEnergy_{\GrainBoundary} / \InterfaceEnergy_{\Surface} = \num{0.1}$}
        \label{fig:two_particle/studies/surface_boundary_energy/0.10000/evolution}
    \end{subfigure}
    \begin{subfigure}{\linewidth}
        \centering
        \includegraphics[width=\linewidth]{sim/two_particle/studies/surface_boundary_energy/0.90000/evolution}
        \caption{$\InterfaceEnergy_{\GrainBoundary} / \InterfaceEnergy_{\Surface} = \num{0.9}$}
        \label{fig:two_particle/studies/surface_boundary_energy/0.90000/evolution}
    \end{subfigure}
    \caption{Geometry Evolutions for Selected Surface-Boundary Energy Ratios}
    \label{fig:two_particle/studies/surface_boundary_energy/evolution}
\end{figure}

\begin{figure}
    \begin{subfigure}{\linewidth}
        \centering
        \includegraphics[width=\linewidth]{sim/two_particle/studies/surface_boundary_energy/neck_size}
        \caption{Evolution Curves}
        \label{fig:two_particle/studies/surface_boundary_energy/neck_size}
    \end{subfigure}
    \begin{subfigure}{\linewidth}
        \centering
        \includegraphics[width=\linewidth]{sim/two_particle/studies/surface_boundary_energy/neck_size_map}
        \caption{Time Isolines}
        \label{fig:two_particle/studies/surface_boundary_energy/neck_size_map}
    \end{subfigure}
    \caption{Neck Size in Dependence on Surface-Boundary Energy Ratio}
\end{figure}

\begin{figure}
    \begin{subfigure}{\linewidth}
        \centering
        \includegraphics[width=\linewidth]{sim/two_particle/studies/surface_boundary_energy/shrinkage}
        \caption{Evolution Curves}
        \label{fig:two_particle/studies/surface_boundary_energy/shrinkage}
    \end{subfigure}
    \begin{subfigure}{\linewidth}
        \centering
        \includegraphics[width=\linewidth]{sim/two_particle/studies/surface_boundary_energy/shrinkage_map}
        \caption{Time Isolines}
        \label{fig:two_particle/studies/surface_boundary_energy/shrinkage_map}
    \end{subfigure}
    \caption{Shrinkage in Dependence on Surface-Boundary Energy Ratio}
\end{figure}

This study investigates the effect of the ratio between grain boundary and surface energy.
Low values mean a high driving force for generating a grain boundary in favor of surface consumption.
Values near \num{1.0} mean there is no energetic benefit of creating a grain boundary and any driving force is purely of geometric nature.
In reality, usually $\InterfaceEnergy_{\GrainBoundary} / \InterfaceEnergy_{\Surface} < \num{0.5}$ for metallic and ceramic substances \cite{Exner1978}.
The initial particle geometry and the material parameters are taken equal on both particles.

\Cref{fig:two_particle/studies/surface_boundary_energy/evolution} shows two selected evolutions of the contact geometry.
Note the large dihedral angle between the surfaces at low interface energy ratios and the small one at high ratios, since in the latter case the local stationary state around the neck is reached earlier.
When $\InterfaceEnergy_{\GrainBoundary} / \InterfaceEnergy_{\Surface} \rightarrow \num{1}$ the dihedral angle approaches \ang{120} as is proposed by \cref{eq:youngs-dihedral}.
Accordingly, when $\InterfaceEnergy_{\GrainBoundary} / \InterfaceEnergy_{\Surface} \rightarrow \num{0}$ the dihedral angle approaches \ang{180}.
With high energy ratio (large dihedral angle), the particle surface around the neck has a smooth concavely curved shape.
With low energy ratio (small dihedral angle) the surface near the neck appears almost flat with transition into the convex surface on the backside of the particles.
The large dihedral angle also corresponds with large achievable neck sizes.

\Cref{fig:two_particle/studies/surface_boundary_energy/neck_size} shows the dependence of the neck size on the interface energy ratio.
The neck size decreases notably with the energy ratio.
A larger decrease appears as the energy ratio approaches \num{2}.
The main influences suspected here are the decrease in driving force (decrease of $\InterfaceEnergy_{\GrainBoundary} / \InterfaceEnergy_{\Surface}$) in conjunction with the geometric effect of the smaller dihedral angle which have additive effect.
As $\InterfaceEnergy_{\GrainBoundary} / \InterfaceEnergy_{\Surface} \rightarrow \num{2}$, the driving force vanishes completely, as there is no energetic benefit in growing the neck in all possible geometric configurations.

\Cref{fig:two_particle/studies/surface_boundary_energy/shrinkage} shows the dependence of the shrinkage on the interface energy ratio.
The dependence of shrinkage on the energy ratio is similar to that of neck size.
In addition to the previously described driving force effect, there is a counter-acting influence of a longer diffusion path and a larger volume displacement needed, when the neck is larger.

These findings are in accordance to several other studies on the influence of the grain boundary energy ratio such as \cite{Pask1975, Cannon1989, Kellett1989, Lange1989, Delannay2015, Gouvea2024}.

\subsubsection{Asymmetric Surface Diffusion Ratio $\DiffusionCoefficient_{\Surface2} / \DiffusionCoefficient_{\Surface1}$}\label{subsubsec:parameter-study-asymmetric-surface-diffusion-ratio}

\begin{figure}[p]
    \begin{subfigure}{0.5\linewidth}
        \includegraphics[scale=0.5]{sim/two_particle/studies/surface_diffusion_asymmetric/shrinkage}
        \caption{Evolution Curves}
        \label{fig:two_particle/studies/surface_diffusion_asymmetric/shrinkage}
    \end{subfigure}%
    \begin{subfigure}{0.5\linewidth}
        \includegraphics[scale=0.5]{sim/two_particle/studies/surface_diffusion_asymmetric/shrinkage_map}
        \caption{Isolines}
        \label{fig:two_particle/studies/surface_diffusion_asymmetric/shrinkage_map}
    \end{subfigure}
    \caption{Shrinkage in Dependence on Asymmetric Surface Diffusion Ratio}
\end{figure}

\begin{figure}[p]
    \begin{subfigure}{0.5\linewidth}
        \includegraphics[scale=0.5]{sim/two_particle/studies/surface_diffusion_asymmetric/neck_size}
        \caption{Evolution Curves}
        \label{fig:two_particle/studies/surface_diffusion_asymmetric/neck_size}
    \end{subfigure}%
    \begin{subfigure}{0.5\linewidth}
        \includegraphics[scale=0.5]{sim/two_particle/studies/surface_diffusion_asymmetric/neck_size_map}
        \caption{Isolines}
        \label{fig:two_particle/studies/surface_diffusion_asymmetric/neck_size_map}
    \end{subfigure}
    \caption{Neck Size in Dependence on Asymmetric Surface Diffusion Ratio}
\end{figure}

\begin{figure}[p]
    \begin{subfigure}{0.5\linewidth}
        \includegraphics[scale=0.5]{sim/two_particle/studies/surface_diffusion_asymmetric/10.00000/evolution}
        \caption{$\DiffusionCoefficient_{\Surface2} / \DiffusionCoefficient_{\Surface1} = \num{10}$}
        \label{fig:two_particle/studies/surface_diffusion_asymmetric/10.00000/evolution}
    \end{subfigure}%
    \begin{subfigure}{0.5\linewidth}
        \includegraphics[scale=0.5]{sim/two_particle/studies/surface_diffusion_asymmetric/100.00000/evolution}
        \caption{$\DiffusionCoefficient_{\Surface2} / \DiffusionCoefficient_{\Surface1} = \num{100}$}
        \label{fig:two_particle/studies/surface_diffusion_asymmetric/100.00000/evolution}
    \end{subfigure}
    \caption{Geometry Evolutions for Selected Asymmetric Surface Diffusion Ratios}
\end{figure}

\subsection{Asymmetric Surface Energy Ratio \texorpdfstring{$\InterfaceEnergy_{\Surface2} / \InterfaceEnergy_{\Surface1}$}{γS2/γS1}}\label{subsec:parameter-study-asymmetric-surface-energy-ratio}

\begin{figure}[p]
    \begin{subfigure}{0.5\linewidth}
        \includegraphics[scale=0.5]{sim/two_particle/studies/surface_energy_asymmetric/2.00000/evolution}
        \caption{$\InterfaceEnergy_{\Surface2} / \InterfaceEnergy_{\Surface1} = \num{2}$}
        \label{fig:two_particle/studies/surface_energy_asymmetric/2.00000/evolution}
    \end{subfigure}%
    \begin{subfigure}{0.5\linewidth}
        \includegraphics[scale=0.5]{sim/two_particle/studies/surface_energy_asymmetric/3.00000/evolution}
        \caption{$\InterfaceEnergy_{\Surface2} / \InterfaceEnergy_{\Surface1} = \num{3}$}
        \label{fig:two_particle/studies/surface_energy_asymmetric/3.00000/evolution}
    \end{subfigure}
    \caption{Geometry Evolutions for Selected Asymmetric Surface Energy Ratios}
    \label{fig:two_particle/studies/surface_energy_asymmetric/evolution}
\end{figure}

\begin{figure}[p]
    \begin{subfigure}{0.5\linewidth}
        \includegraphics[scale=0.5]{sim/two_particle/studies/surface_energy_asymmetric/neck_size}
        \caption{Evolution Curves}
        \label{fig:two_particle/studies/surface_energy_asymmetric/neck_size}
    \end{subfigure}%
    \begin{subfigure}{0.5\linewidth}
        \includegraphics[scale=0.5]{sim/two_particle/studies/surface_energy_asymmetric/neck_size_map}
        \caption{Time Isolines}
        \label{fig:two_particle/studies/surface_energy_asymmetric/neck_size_map}
    \end{subfigure}
    \caption{Neck Size in Dependence on Asymmetric Surface Energy Ratio}
\end{figure}

\begin{figure}[p]
    \begin{subfigure}{0.5\linewidth}
        \includegraphics[scale=0.5]{sim/two_particle/studies/surface_energy_asymmetric/shrinkage}
        \caption{Evolution Curves}
        \label{fig:two_particle/studies/surface_energy_asymmetric/shrinkage}
    \end{subfigure}%
    \begin{subfigure}{0.5\linewidth}
        \includegraphics[scale=0.5]{sim/two_particle/studies/surface_energy_asymmetric/shrinkage_map}
        \caption{Time Isolines}
        \label{fig:two_particle/studies/surface_energy_asymmetric/shrinkage_map}
    \end{subfigure}
    \caption{Shrinkage in Dependence on Asymmetric Surface Energy Ratio}
\end{figure}

This study investigates the effect of sintering particles of different substance by means of a difference in their surface energy.
The initial particle geometry and remaining material parameters are taken equal on both particles with a grain boundary energy of $\InterfaceEnergy_{\GrainBoundary} = \num{0.5}\InterfaceEnergy_{\Surface1}$.
The study is limited to ratios up to $\InterfaceEnergy_{\Surface2}/\InterfaceEnergy_{\Surface1} = \num{3}$, because above this point the model showed numerical instability.
At high surface energy ratios an undercut forms on the lower surface energy particle's surface for thermodynamic reasons as discussed below.
Due to the mathematical construction of the model, volume can be removed from a surface node in such an undercut despite there is too few left between the surface and the grain boundary, so after a critical time step, surface line and grain boundary cross.
This is of course physically impossible, but at the time of writing this thesis, the issue was not solved.

\Cref{fig:two_particle/studies/surface_energy_asymmetric/evolution} shows two selected evolutions of the contact geometry.
Note especially the undercut occurring near the neck on the surface of particle 1.
Due to the high surface energy of the second particle it is thermodynamical beneficial to grow the neck while creating additional surface of the first particle.
As the surface diffusion is to slow to deliver enough volume to the neck, an undercut evolves.
This process is similar to the relation between surface energy and grain boundary in general determining the dihedral angle (see \cref{subsec:parameter-study-surface-boundary-energy}) with the main difference that here three different interface energies are in action leading to an asymmetric neck geometry.
Young's equation (\cref{eq:youngs-dihedral}) only works here for $\InterfaceEnergy_{\Surface2}/\InterfaceEnergy_{\Surface1} < \num{1.5}$ as for higher ratios equilibrium is not possible (see \cref{sec:diffusion}).
Remarkably that for $\num{1} < \InterfaceEnergy_{\Surface2}/\InterfaceEnergy_{\Surface1} < \num{1.5}$ the equilibrium dihedral angle was present in later stages in the results, but the angles between surfaces and grain boundaries did not fit to those proposed by \cref{eq:youngs-dihedral}.
So for example the simulation for $\InterfaceEnergy_{\Surface2}/\InterfaceEnergy_{\Surface1} = \num{1.25}$ showed a dihedral angle $\DihedralAngle \approx \ang{158}$ which is in coincidence with Young's equation proposal, but the simulated angles to the grain boundaries were \ang{96} (left) and \ang{106} (right) compared to the proposed \ang{72} and \ang{130}.
The main reason for this is suspected in the coarse discretization of the grain boundary by a single node (see \cref{sec:contact-conditions} for the reasoning behind this).
When imagining a smoothly curved grain boundary, the fit would be closer.

\Cref{fig:two_particle/studies/surface_energy_asymmetric/neck_size} shows the dependence of the neck size on the asymmetric surface energy ratio.
Neck size increases slightly with the surface energy ratio, as it becomes more beneficial to grow the neck in favor of lowering the second particle's surface, although more surface of the first particle is created.

\Cref{fig:two_particle/studies/surface_energy_asymmetric/shrinkage} shows the dependence of the shrinkage on the asymmetric surface energy ratio.
Shrinkage increases slightly with the surface energy ratio, too, since the driving force for grain boundary diffusion of the second particle is increased.
Also, the neck undercut creates a driving force for the first particle to deliver more volume to the undercut neck region, which is partially served by diffusion from the grain boundary.

