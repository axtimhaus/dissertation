\addchap{Summary}\label{ch:summary}

\minisec{Deutsche Version}
Es wurde ein neuartiges Modell der Sinterung zweier bzw.~weniger Teilchen unter Verwendung scharfer Grenzflächen entwickelt und implementiert.
Die Anwendung des thermodynamischen Extremalprinzips erlaubt eine elegante und erweiterbare Formulierung.
Das numerische Verhalten des Modells unter Anwendung von Neuvernetzungsroutinen wurde untersucht und bewertet.
Es wurden Parameterstudien zu klassischen dimensionslosen Kenngrößen von Zweiteilchenmodellen, sowie asymmetrischer Material- wie Geometriepaarung durchgeführt, mit Literaturwissen abgeglichen und erklärt.
Kontakte mehrerer Teilchen wurden auf den Einfluss der Porenschließung untersucht, auch bei Anwesenheit inerter Teilchen.
Die Formcharakteristik von Pulverteilchen wurde unter statistischer Anwendung einer Formfunktion beschrieben und zur Generierung von Pulverteilchen als Eingabe in die Simulation verwendet.
Das mittlere Verhalten der zufällig generierten Teilchen wurde zur Beschreibung des Verhaltens des Pulvers genutzt und Unterschiede zu klassischen Verfahren der Mittelung diskutiert.

\minisec{English Version}
A novel model for the sintering of two or a couple of particles using sharp interfaces was developed and implemented.
The application of the thermodynamic extremal principle allows a concise and extendable formulation.
The numerical behavior of the model was investigated and evaluated under application of remeshing routines.
Parameter studies were conducted on classical dimensionless parameters of two-particle models, as well as asymmetric material and geometry pairing, which were compared with literature knowledge and explained.
Contacts between multiple particles were investigated for the influence of pore closure, even in the presence of inert particles.
The shape characteristics of powder particles were described using a statistical shape function and used to generate powder particles as input for the simulation.
The average behavior of the randomly generated particles was used to describe the behavior of the powder, and differences from classical averaging methods were discussed.
