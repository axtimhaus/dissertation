\section{Architecture of the Simulation Program}\label{sec:architecture}

The model is implemented in a C\# program library including
\begin{itemize}
    \item data structures that represent sintering processes, particles, nodes and their properties
    \item a planar geometry library based on hierarchically defined coordinate systems
    \item a solver routine with adjustable subroutines for the diffusion problem as described in \cref{sec:solution}
    \item remeshing routines as described in \cref{sec:remeshing-procedures}
    \item basic algorithms for contact determination and creation (not described here)
    \item some more infrastructure (e.g.~data storage, plotting, \ldots).
\end{itemize}
The package is designed with extensibility in mind.
Therefore, most subroutines like remeshing, integration scheme and step estimation can be selected or replaced by the user if necessary.
The user can build a simulation by
\begin{itemize}
    \item defining the initial state of the particle system
    \item defining one or multiple processing steps at constant temperature (stepwise constant temperature history approximation)
    \item selecting quantities (state variables and fluxes) to be evolved in the system as well as respective constraints or use the default set
    \item selecting the numerical routines or use the default set.
\end{itemize}
The code is made available to the public~\cite{RefraSin1.0}.
