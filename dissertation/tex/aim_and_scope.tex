\chapter{Aim and Scope}\label{ch:aim-and-scope}

This work is intended to create and evaluate a novel model approach for sintering of irregularly shaped particles.
The model is based on a \emph{finite differences} (FD) approach of particle description, thus the interfaces are described as sharp lines.
FOr the solution of the complicated partial differential equation system describing the diffusional flows and geometry evolution, the \emph{thermodynamic extremal principle} (TEP) shall be applied.
The model shall be restricted to 2D-space as a proof of concept, but may be extended to 3D-space as well.
The latter is out of scope of this work.

The work resides under the circumstances of the RefraBund project as described in \autoref{sec:project-group}\@.
Therefore, the following influences shall be included in the model and investigated by simulation as well experiment.
\begin{description}
    \item[Multi-Material Powder] The investigated material consists of aluminium oxide powder as well as metallic niobium resp.~tantalum powder.
    The model must therefore be able to respect different material properties on particles in contact to each other with their implication of grain boundary shape and kinetic behavior.
    \item[Multi-Scale Powder] For achieving good thermal shock properties a component shall consist of coarse-grained material with fine-grained additions as filler and bonder.
    The model must be able to deal with particle sizes that differ by several orders of magnitude, which is mainly a topic of computational effort and numerical stability.
    \item[Aggregate Powders] The coarse grained powder is obtained be sintering of fine-grained composite powders to produce aggregate particles.
    The aggregate particles are not homogeneous in their properties and include a certain porosity.
    The model must include contacts of aggregate particles by average approaches to circumvent the need of local structure description and modelling.
    \item[Powder Mixture] The production process involves mixing of different powder fractions to obtain optimal properties in processing and application.
    The mixing shall be modelled by statistical approaches to enable the investigation of different powder mixture to support product development.
\end{description}

The predictions of the model are characterized in this work in comparison to other models.
The model is validated against sintering experiments under consideration of microstructure and physical material properties.